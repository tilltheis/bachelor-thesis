%!TEX root = thesis.tex

\chapter{Vorbereitung} % (fold)
\label{cha:anhang_vorbereitung}

In diesem Anhang geht es darum, wie die Entwicklungsumgebung aufgesetzt wird, mit welcher Version der eingesetzten Software gearbeitet wird und wie sich das Play-Framework installieren lässt.

\section{Entwicklungsumgebung} % (fold)
\label{sec:entwicklungsumgebung}

Zum Anlegen von Play-Projekten und Steuern des Web-Servers wird eine Kommandozeile, wie die Eingabeaufforderung unter Windows oder dem Terminal und Mac OS~X benötigt.
Zum Programmieren kann entweder ein einfacher Text-Editor, oder auch eine IDE verwendet werden.
Für die einzelnen Programme können teilweise Plugins heruntergeladen werden, die beispielsweise Code Completion oder Syntax Highlighting für Play-spezifische Funktionalitäten bereitstellen.

Für \textbf{Eclipse} kann die Scala IDE von \citealt{scala_ide} heruntergeladen werden.
Diese IDE ist für die Entwicklung von Scala-Anwendungen mit Eclipse notwendig.
Für die Scala IDE existiert außerdem ein Plugin für das Play-Framework.
Eine genaue Installationsanleitung dafür ist unter \citealt{scala_ide_play_plugin} zu finden.
Um ein Play-Projekt in Eclipse zu bearbeiten, muss dafür nach dem Anlegen des Projekts auf der Kommandozeile ein Eclipse-Projekt exportiert werden.
Dazu muss in der Play-Konsole des Projekts der Befehl \lstinline|eclipse| ausgeführt werden.
Dieser Befehl muss jedes Mal ausgeführt werden, wenn die Konfigurationsdateien geändert werden.
Anschließend kann der Ordner des Play-Projekts als Eclipse-Projekt importiert und geöffnet werden.

Beim Exportieren eines \textbf{IntelliJ IDEA}-Projekts verhält es sich ähnlich, wie im Falle von Eclipse.
Der einzige Unterschied ist, dass der Befehel \lstinline|idea| verwendet werden muss, um ein IntelliJ IDEA-Projekt zu erstellen.
Für Text-Editoren, wie z.B. \textbf{VIM}, \textbf{Emacs} oder \textbf{Sublime Text} ist prinzipiell kein Plugin nötig.
Falls aber ein Plugin für den verwendeten Editor existiert, kann es natürlich installiert werden.
Weitere Informationen zu diesen und \textbf{anderen Entwicklungsumgebungen} sind in der offiziellen Dokumentation zu finden \cite[vgl.][]{ide}.

% section entwicklungsumgebung (end)


\section{Software-Version} % (fold)
\label{sec:software_version}

Zum Zeitpunkt dieser Arbeit ist die aktuellste stabile Version des Play-Frameworks die Version 2.2.0.
Diese Version wird in allen Beispielen und der eigenen entwickelten Anwendung verwendet.
Für den Einsatz des Frameworks wird das JDK in Version 6 oder 7 benötigt.
Der Autor arbeitet mit dem JDK~7 unter Mac OS~X.
Die aktuell verwendete Scala-Version ist 2.10.2, diese ist unabhängig von der systemweit installierten Version und wird beim Anlegen eines Play-Projekts für das jeweilige Projekt installiert.

% section software_version (end)


\section{Installation} % (fold)
\label{sec:installation}

Zum Installieren von Play kann eine vorkompilierte Version der Software von der offiziellen Website unter \cite{play_download} heruntergeladen werden.
Nach dem Entpacken der heruntergeladenen Datei kann das Programm namens \lstinline|play|, das sich im entpackten Ordner befindet, auf der Kommandozeile verwendet werden.
Um für die Verwendung des Programms nicht immer den absoluten Pfad angeben zu müssen, kann der Pfad zum entpackten Ordner in die Systemumgebungsvariable PATH eingetragen werden.

Unter \textbf{Mac OS~X} muss dazu auf der Kommandozeile folgender Befehl ausgeführt werden: \lstinline|export PATH=$PATH:<Pfad zum entpackten Ordner>|.
Ein konkretes Beispiel könnte so aussehen: \lstinline|export PATH=$PATH:~/Downloads/play-2.1.3|.
Damit sich der Ordner auch nach dem Schließen des Terminals noch in der \lstinline|PATH|-Variable befindet, sollte der obige Befehl zusätzlich in die \lstinline|~/.profile|-Datei oder in die Konfigurationsdatei der verwendeten Shell eingetragen werden.
Falls diese Datei nicht existiert, muss sie vorher angelegt werden.

Unter \textbf{Windows} muss dazu in der Eingabeaufforderung folgender Befehl ausgeführt werden: \lstinline|setx PATH "%PATH%;c:\path\to\play" /m|, wobei \lstinline|c:\path\to\play| durch den tatsächlichen Pfad zu ersetzen ist \cite[vgl.][S.~9]{play_for_scala}.

% section installation (end)

% chapter anhang_vorbereitung (end)


\chapter{Inhalt der beiliegenden CD} % (fold)
\label{cha:inhalt_der_beiliegenden_cd}

Dieser Arbeit liegt ein Datenträger mit folgender Verzeichnisstruktur bei:

\begin{description}[leftmargin=!,labelwidth=\widthof{\bfseries /age\_statistics\_http/}]
  \item[/thesis.pdf] dieses Dokument im PDF-Format
  \item[/age\_statistics\_http/] die in Kap.~\ref{cha:grundlagen} entwickelte Altersstatistikenanwendung
  \item[/age\_statistics\_sse/] die in Kap.~\ref{cha:grundlagen} entwickelte Anwendung mit den in Unterabschnitt~\ref{sub:server_sent_events_in_der_altersstatistiken_anwendung} eingeführten Server Sent Events-Erweiterungen
  \item[/age\_statistics\_ws/] die in Kap.~\ref{cha:grundlagen} entwickelte Anwendung mit den in Unterabschnitt~\ref{sub:web_sockets_in_der_altersstatistiken_anwendung} eingeführten Web Sockets-Erweiterungen
  \item[/examples/] SBT-Projekt \cite[vgl.][]{sbt}, das den in Kap.~\ref{cha:reaktive_programmierung} gezeigten Code enthält (inklusive automatisierter Tests)
  \item[/twitter\_news/] die in Kap.~\ref{cha:anwendung} entwickelte Twitter News-Anwendung
\end{description}

% chapter inhalt_der_beiliegenden_cd (end)