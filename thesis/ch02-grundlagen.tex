%!TEX root = thesis.tex

\chapter{Grundlagen} % (fold)
\label{cha:grundlagen}

Grundlagen mit den bisherigen Quellen \citealt{play_for_scala_v8}.

In diesem Kapitel werden die grundlegenden Techniken zur Webseiten-Entwicklung mit dem Play-Framework vorgestellt werden.
Dabei wird dafür benötigtes Hintergrundwissen bereitgestellt und erklärt, welche Werkzeuge benötigt werden.
Schließlich werden durch die Entwicklung einer kleinen Anwendung die einzelnen Komponenten erklärt.
Es werden hierbei in erster Linie die Komponenten vorgestellt, die für die Entwicklung von Real-Time-Web-Anwendungen unbedingt notwendig sind.


\section{Vorbereitung} % (fold)
\label{sec:vorbereitung}

Bevor auf die eigentliche Arbeit mit dem Play-Framework eingegangen wird, müssen einige vorbereitende Dinge geklärt werden. In diesem Abschnitt geht es darum, wie die Entwicklungsumgebung aufgesetzt wird, mit welcher Version der eingesetzten Software gearbeitet wird und wie sich das Play-Framework installieren lässt.

\subsection{Entwicklungsumgebung} % (fold)
\label{sub:entwicklungsumgebung}

Zum Anlegen von Play-Projekten und Steuern des Web-Servers wird eine Kommandozeile, wie die Eingabeaufforderung unter Windows oder dem Terminal und Mac OS~X benötigt.
Zum Programmieren kann entweder ein einfacher Text-Editor, oder auch eine IDE (integrated development environment) verwendet werden.
Für die einzelnen Programme können teilweise Plugins heruntergeladen werden, die beispielsweise code completion oder syntax highlighting für Play-spezifische Funktionalitäten bereitstellen.

Für \textbf{Eclipse} kann die Scala IDE von \citealt{scala_ide} heruntergeladen werden.
Diese IDE ist für die Entwicklung von Scala-Anwendungen mit Eclipse notwendig.
Für die Scala IDE existiert außerdem ein Plugin für das Play-Framework.
Eine genaue Installationsanleitung dafür ist unter \citealt{scala_ide_play_plugin} zu finden.
Um ein Play-Projekt in Eclipse zu bearbeiten, muss dafür nach dem Anlegen des Projekts auf der Kommandozeile ein Eclipse-Projekt exportiert werden.
Dazu muss in der Play-Konsole des Projekts der Befehl \lstinline|eclipse| ausgeführt werden.
Dieser Befehl muss jedes Mal ausgeführt werden, wenn die Konfigurationsdateien geändert werden.
Anschließend kann der Ordner des Play-Projekts als Eclipse-Projekt importiert und geöffnet werden.

Beim Exportieren eines \textbf{IntelliJ IDEA}-Projekts verhält es sich ähnlich, wie im Falle von Eclipse.
Der einzige Unterschied ist, dass der Befehel \lstinline|idea| verwendet werden muss, um ein IntelliJ IDEA-Projekt zu erstellen.
Für Text-Editoren, wie z.B. \textbf{VIM}, \textbf{Emacs} oder \textbf{Sublime Text} ist prinzipiell kein Plugin nötig, es kann aber eines installiert werden, sofern eines vorhanden ist.
Weitere Informationen zu diesen und \textbf{anderen Entwicklungsumgebungen} sind in der offiziellen Dokumentation zu finden \cite[vgl.][]{ide}.

% subsection entwicklungsumgebung (end)


\subsection{Software-Version} % (fold)
\label{sub:software_version}

Zum Zeitpunkt dieser Arbeit ist die aktuellste stabile Version des Play-Frameworks die Version 2.1.2.
Diese Version wird in allen Beispielen und der eigenen entwickelten Anwendung verwendet.
Für den Einsatz des Frameworks wird das JDK (Java development kit) in Version 6 oder 7 benötigt.
Der Autor arbeitet mit dem JDK~7 unter Mac OS~X.

% subsection software_version (end)


\subsection{Installation} % (fold)
\label{sub:installation}

Zum Installieren von Play kann eine vorkompilierte Version der Software von der offiziellen Website unter \cite{play_download} heruntergeladen werden.
Nach dem Entpacken der heruntergeladenen Datei kann das Programm namens \lstinline|play|, das sich im entpackten Ordner befindet, auf der Kommandozeile verwendet werden.
Um für die Verwendung des Programms nicht immer den absoluten Pfad angeben zu müssen, kann der Pfad zum entpackten Ordner in die Systemumgebungsvariable PATH eingetragen werden.

Unter \textbf{Mac OS~X} muss dazu auf der Kommandozeile folgender Befehl ausgeführt werden: \lstinline|export PATH=$PATH:<Pfad zum entpackten Ordner>|.
Ein konkretes Beispiel könnte so aussehen: \lstinline|export PATH=$PATH:~/Downloads/play-2.1.2|.
Damit sich der Ordner auch nach dem Schließen des Terminals noch in der PATH-Variable befindet, sollte der obige Befehl zusätzlich in die \lstinline|~/.profile|-Datei eingetragen werden.
Falls diese Datei nicht existiert, muss sie vorher angelegt werden.

Unter \textbf{Windows} muss dazu in der Eingabeaufforderung folgender Befehl ausgeführt werden: \lstinline|setx PATH "%PATH%;c:\path\to\play" /m|, wobei \lstinline|c:\path\to\play| durch den tatsächlichen Pfad zu ersetzen ist \cite[vgl.][S.~9]{play_for_scala_v8}.

% subsection installation (end)


% section vorbereitung (end)


\section{Architektur} % (fold)
\label{sec:architektur}

Auf der untersten Ebene existiert ein Web-Server, der mit dem Framework ausgeliefert wird.
Die Anfragen, die der Web-Server empfängt werden an Play weitergeleitet und schließlich von der Anwendung verarbeitet.
Nach der Verarbeitung wird eine Antwort generiert und schließlich als HTTP-Antwort versendet.
Der Anwendungscode einer Play-Applikation ist nach der \gls{mvc}-Architektur aufgebaut \cite[vgl.][S.~51-53]{play_for_scala_v8}.

"`[\gls{mvc} programming is a] three-way factoring, whereby objects of different classes take over the operations related to the application domain (the model), the display of the application's state (the view), and the user interaction with the model and the view (the controller)."' \cite[vgl.][S.~1]{mvc}.
Views bilden Models i.~d.~R. auf HTML-Seiten ab.
Controller sind dazu da, HTTP-Anfragen zu verarbeiten, die jeweilige Antwort durch eine View übersetzen zu lassen und als HTTP-Antwort zurückzusenden.

% section architektur (end)


% chapter grundlagen (end)