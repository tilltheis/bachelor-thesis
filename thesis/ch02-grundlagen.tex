%!TEX root = thesis.tex

\chapter{Grundlagen} % (fold)
\label{cha:grundlagen}

Grundlagen mit den bisherigen Quellen \citealt{hilton2013}.


\section{Vorbereitung} % (fold)
\label{sec:vorbereitung}

\subsection{Entwicklungsumgebung} % (fold)
\label{sub:entwicklungsumgebung}

Es kann entweder mit einem einfachen Texteditor entwickelt werden, oder auch mit einer IDE (Integrated development environment).

Beispiel Sublime Text 2.

Als IDE wird hier IntelliJ IDEA 12 Community Edition verwendet.
Nach dem Anlegen des Play-Projekts auf der Kommandozeile muss ein mal der Befehl \lstinline|play idea| ausgeführt werden.
Wenn man sich bereits in der Play-Eingabemaske befindet, muss man nur \lstinline|idea| ausführen.
Eine umfangreiche Liste für weitere IDEs ist bei \citealt{ide}~\footnote{\url{http://www.playframework.com/documentation/2.1.1/IDE}} zu finden.

% subsection entwicklungsumgebung (end)

% section vorbereitung (end)


% chapter grundlagen (end)