%!TEX root = thesis.tex

% vgl. http://tex.stackexchange.com/questions/8946/how-to-combine-acronym-and-glossary

% \newglossaryentry{computer}{name=computer, description={is the rate of deactivation from ... and emission)}, sort=k}

% \newacronym{lvm}{LVM}{Logical Volume Manager}

% \newglossaryentry{lvm}{name=LVM, description={a Logical Volume Manager (LVM) does smth fancy}, first={Logical Volume Manager (LVM)}}



\newglossaryentry{adt}
{
  name        = algebraischer Datentyp,
  description = {ein algebraischer Datentyp ist ein aus anderen Typen zusammengesetzter Datentyp, der mittels Pattern Matching zerlegt werden kann},
}


\newglossaryentry{kovarianz}
{
  name        = Kovarianz,
  description = {ist ein Datentyp \lstinline|T| über einen kovarianten Typ \lstinline|A| parametrisiert, so ist \lstinline|T[B]| ein Sub-Typ von \lstinline|T[A]|, falls \lstinline|B| ein Sub-Typ von \lstinline|A| ist},
}


\newglossaryentry{nothing}
{
  name        = Nothing,
  description = {Der Scala-Typ \lstinline|Nothing| ist eine Spezialisierung jedes anderen Typs und kann deshalb jeden anderen Typ annehmen \cite[vgl.][S.~216]{programming_in_scala}},
}


\newglossaryentry{json}
{
  name        = JSON,
  description = {Mit der JavaScript Object Notation (JSON) lassen sich JavaScript-Objekte als literale notieren},
  first       = {JavaScript Object Notation (JSON)}
}