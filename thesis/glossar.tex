%!TEX root = thesis.tex

% vgl. http://tex.stackexchange.com/questions/8946/how-to-combine-acronym-and-glossary

% \newglossaryentry{computer}{name=computer, description={is the rate of deactivation from ... and emission)}, sort=k}

% \newacronym{lvm}{LVM}{Logical Volume Manager}

% \newglossaryentry{lvm}{name=LVM, description={a Logical Volume Manager (LVM) does smth fancy}, first={Logical Volume Manager (LVM)}}



\newglossaryentry{mvc}
{
  name        = MVC,
  description = {"`[The Model-View-Controller (MVC) architecture is a] three-way factoring, whereby objects of different classes take over the operations related to the application domain (the model), the display of the application's state (the view), and the user interaction with the model and the view (the controller)."' \cite[vgl.][S.~1]{mvc}},
  first       = {Model-View-Controller (MVC)}
}


\newglossaryentry{js}
{
  name        = JS,
  description = {JavaScript (JS) ist eine Programmiersprache für clientseitige Programmierung im Browser},
  first       = {JavaScript (JS)}
}


\newglossaryentry{css}
{
  name        = CSS,
  description = {Cascading Stylesheets (CSS) ist eine Sprache mit der sich das Aussehen von Websites beeinflussen lässt},
  first       = {Cascading Stylesheets (CSS)}
}


\newglossaryentry{ide}
{
  name        = IDE,
  description = {eine IDE (integrated development environment) ist ein Programm, das unteschiedliche Komponenten, die zur Softwareentwicklung wichtig sind, vereint},
  first       = {IDE (integrated development environment)}
}


\newglossaryentry{adt}
{
  name        = ADT,
  description = {ein algebraischer Datentyp (ADT) ist ein aus anderen Typen zusammengesetzter Datentyp, der mittels Pattern Matching zerlegt werden kann},
  first       = {algebraischer Datentyp (ADT)}
}


\newglossaryentry{router}
{
  name        = Router,
  description = {der Router einer Play-Anwendung bildet URLs auf Controller-Aktionen ab},
}


\newglossaryentry{jdk}
{
  name        = JDK,
  description = {das JDK (Java Development Kit) enthält die Java-Laufzeitumgebung und einige Entwicklungswerkzeuge},
}


\newglossaryentry{kovarianz}
{
  name        = Kovarianz,
  description = {ist ein Datentyp \lstinline|T| über einen kovarianten Typ \lstinline|A| parametrisiert, so ist \lstinline|T[B]| ein Sub-Typ von \lstinline|T[A]|, falls \lstinline|B| ein Sub-Typ von \lstinline|A| ist},
}


\newglossaryentry{nothing}
{
  name        = Nothing,
  description = {Der Scala-Typ \lstinline|Nothing| ist eine Spezialisierung jedes anderen Typs \cite[vgl.][S.~32]{scala_specification}},
}


\newglossaryentry{continuation}
{
  name        = Continuation,
  description = {Eine Delimited Continuation repräsentiert den Rest einer Berechnung bis zu einem bestimmten Punkt \cite[vgl.][S.~1]{continuations}},
}