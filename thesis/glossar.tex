%!TEX root = thesis.tex

% vgl. http://tex.stackexchange.com/questions/8946/how-to-combine-acronym-and-glossary

% \newglossaryentry{computer}{name=computer, description={is the rate of deactivation from ... and emission)}, sort=k}

% \newacronym{lvm}{LVM}{Logical Volume Manager}

% \newglossaryentry{lvm}{name=LVM, description={a Logical Volume Manager (LVM) does smth fancy}, first={Logical Volume Manager (LVM)}}



\newglossaryentry{mvc}
{
  name        = MVC,
  description = {"`[The Model-View-Controller (MVC) architecture is a] three-way factoring, whereby objects of different classes take over the operations related to the application domain (the model), the display of the application's state (the view), and the user interaction with the model and the view (the controller)."' \cite[vgl.][S.~1]{mvc}},
  first       = {Model-View-Controller (MVC)}
}


\newglossaryentry{adt}
{
  name        = algebraischer Datentyp,
  description = {ein algebraischer Datentyp ist ein aus anderen Typen zusammengesetzter Datentyp, der mittels Pattern Matching zerlegt werden kann},
}


\newglossaryentry{router}
{
  name        = Router,
  description = {der Router einer Play-Anwendung bildet URLs auf Controller-Aktionen ab},
}


\newglossaryentry{kovarianz}
{
  name        = Kovarianz,
  description = {ist ein Datentyp \lstinline|T| über einen kovarianten Typ \lstinline|A| parametrisiert, so ist \lstinline|T[B]| ein Sub-Typ von \lstinline|T[A]|, falls \lstinline|B| ein Sub-Typ von \lstinline|A| ist},
}


\newglossaryentry{nothing}
{
  name        = Nothing,
  description = {Der Scala-Typ \lstinline|Nothing| ist eine Spezialisierung jedes anderen Typs und kann deshalb jeden anderen Typ annehmen \cite[vgl.][S.~256--257]{programming_in_scala}},
}


\newglossaryentry{continuation}
{
  name        = Continuation,
  description = {Eine Delimited Continuation repräsentiert den Rest einer Berechnung bis zu einem bestimmten Punkt \cite[vgl.][S.~1]{continuations}},
}


\newglossaryentry{ajax}
{
  name        = Ajax,
  description = {Asynchronous JavaScript and XML (Ajax) ist eine Technik, die asynchrone Kommunikation zwischen Client und Server ermöglicht},
  first       = {Asynchronous JavaScript and XML (Ajax)}
}


\newglossaryentry{json}
{
  name        = JSON,
  description = {Mit der JavaScript Object Notation (JSON) lassen sich JavaScript-Objekte als literale notieren},
  first       = {JavaScript Object Notation (JSON)}
}