%!TEX root = thesis.tex

\chapter{Fazit und Ausblick} % (fold)
\label{cha:fazit_und_ausblick}

In diesem abschließenden Kapitel wird ein Fazit aus den in dieser Arbeit vorgestellten Inhalten gezogen.
Anschließend wird ein Ausblick gegeben, was in dieser Arbeit nicht behandelt werden konnte, aber was aufbauend auf dieser oder in ihrem thematischen Umfeld für Vorschungsmöglichkeiten bestehen.

\section{Fazit} % (fold)
\label{sec:fazit}

Ziel dieser Arbeit war es, herauszufinden, wie mit Play Web-Anwendungen für das Real-Time-Web entwickelt werden können.
Dazu wurde in den Grundlagen erklärt, wie eine einfache statische Web-Anwendung entwickelt wird.
Diese Anwendung wurde mit den Erkenntnissen der Arbeit später um den Real-Time-Aspekt erweitert.
Der statische Teil ließ sich ohne Schwierigkeiten umsetzen, wohingegen der dynamische Teil komplizierter war.
Der komplizierteste Teil war das Verständnis der Iteratee-Streams und das Wissen um die denen zugrundeliegenden Konzepte, wie Monaden und speziell die \lstinline|Future|-Monade.
Erschwerend hinzu kamen die Entdeckung eines Bugs, wodurch die Implementierung verkompliziert wurde.

Iteratee-Streams bestehen aus den drei Teilen \lstinline|Enumerator|s, \lstinline|Iteratee|s und \lstinline|Enumeratee|s, also den Datenquellen, den Datensenken und den Stream-Transformatoren.
In dieser Arbeit wurde nicht nur gezeigt, wie diese Komponenten angewendet werden, sondern auch, wie eigene \lstinline|Enumerator|s, \lstinline|Iteratee|s und \lstinline|Enumeratee|s erstellt werden können.
Dieser Blick auf die Implementierungsebene ermöglicht aus Sicht des Autors ein wesentlich umfassenderes Verständnis von Plays Iteratee-Streams.

Plays Implementierung der Iteratee-Streams weicht von der ursprünglichen Implementierung Kiselyovs vor allem in dem Punkt ab, dass sie nicht mit beliebigen Monaden sondern nur mit \lstinline|Future|-Monaden arbeitet.
Der Grund hierfür wird sein, dass Play das Ziel hat, durch reaktive Programmierung ohne zu Blockieren möglichst effizient zu sein, weshalb sich die \lstinline|Future|-Monade anbietet, auch wenn im Gegenzug Flexibilität eingebüßt wird.
Dies ist aus Sicht des Autors vertretbar und hat für ihn beim Arbeiten mit Play keine Probleme dargestellt.

Dass durch diese Streams aber auch komplexere Prozesse als die erste Beispielanwendung implementiert werden können, wurde mit der Twitter News-Anwendung gezeigt.
Dabei wurden mit \lstinline|Iteratee|s Daten von der Twitter-API empfangen und verarbeitet.
Mit verknüpften \lstinline|Enumeratee|s wurden weitere Tweets heruntergeladen und zwischengespeichert.
Erst nach vollständiger Verarbeitung wurden diese Daten wieder als \lstinline|Enumerator|s der Außenwelt zur Verfügung gestellt und deren Elemente an die verbundenen Clients gesendet.

Play wird aktiv entwickelt und es wurden während der Entstehung dieser Arbeit mehrere neue Versionen veröffentlicht.
Bei der Aktualisierung von Play 2.1 auf Play 2.2 mussten Teile des Codes angepasst werden, was bei einem solchen Versionssprung zu erwarten ist, bei häufigen Versionsänderungen aber zur Last werden kann.
Außerdem wurde beim Schreiben dieser Arbeit ein Bug entdeckt, der aber in der nächsten Version korrigiert wird.
Durch die aktive Entwicklung und Erweiterung des Frameworks ist Play möglicherweise weniger stabil, als ältere Frameworks, bekanntgewordene Fehler werden aber dadurch auch schnell korrigiert.
Insgesamt bilden die von Play zur Verfügung gestellten Werkzeuge aus Sicht des Autors eine solide Grundlage, um Real-Time-Web-Anwendungen in einem reaktiven Stil zu entwickeln.

% section fazit (end)

\section{Ausblick} % (fold)
\label{sec:ausblick}

In dieser Arbeit wurde das Hauptaugenmerk auf Play, also die Server-Seite gerichtet und die Client-Seite nur oberflächlich behandelt.
Durch Einsatz eines passenden JavaScript-Frameworks könnte der client-seitige Code wesentlich besser strukturiert werden, wodurch sich Anwendungen mit einer komplexeren Client-Seite besser umsetzen ließen.
Eine weitere Arbeit könnte sich auf die Programmierung der Browser-Seite mit z.~B. AngularJS \cite[vgl.][]{angular_js} oder \mbox{Ember.js} \cite[vgl.][]{ember_js} konzentrieren, die mit einem in Play geschriebenen Server-Backend kommuniziert. % \mbox to prevent line break

Des Weiteren hat sich diese Arbeit fast ausschließlich mit den reaktiven Möglichkeiten des Play-Frameworks beschäftigt, weil diese für die Entwicklung von Real-Time-Web-Anwendungen nötig sind.
Auf viele andere Teile von Play wurde gar nicht oder nur wenig eingegangen, weil dies den Rahmen dieser Arbeit gesprengt hätte.
So wurde beispielsweise das Arbeiten mit Datenbanken komplett ignoriert und für Controller wurde nur die Programmierung einfacher Actions erklärt.
Auch in dieser Richtung gibt es also weitere Möglichkeiten für Forschungen.

Play legt großen Wert auf reaktive Programmierung.
Dieser Trend reduziert sich allerdings nicht alleine auf Play, sondern wird auch von Martin Odersky, dem Begründer von Scala, in einer Online-Vorlesung \cite[vgl.][]{principles_of_reactive_programming} gelehrt.
Die in dieser Arbeit vorgestellten reaktiven Konzepte sind nur ein kleiner Teil der Dinge, die reaktive Programmierung ausmachen.
Jonas Bonér, der Mitarbeiter bei Typesafe ist, hat unter Mitwirkung anderer Persönlichkeiten der Scala-Community das Reactive Manifesto \cite[vgl.][]{reactive_manifesto} veröffentlicht, in dem die Kernpunkte reaktiver Programmierung beschrieben werden.
Dieses Thema bietet genügend Stoff für eine weiterführende Arbeit darüber.

% section ausblick (end)

% chapter fazit_und_ausblick (end)