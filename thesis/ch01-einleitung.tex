%!TEX root = thesis.tex

\chapter{Einleitung} % (fold)
\label{cha:einleitung}

In diesem Kapitel wird erläutert, was das Real-Time-Web ausmacht und was mit dem Play-Framework und dieser Arbeit erreicht werden soll.
Es wird ein Überblick über die bisherige Entwicklung des Webs gegeben und differenziert, womit sich diese Arbeit genau beschäftigt und womit nicht.
Anschließend wird beschrieben, wie die weitere Struktur dieser Arbeit aussieht.


\section{Entwicklung des Webs} % (fold)
\label{sec:entwicklung_des_webs}

Nach \citealt{evolution_of_the_web}~(S.~1--3) war das World Wide Web der 1990er Jahre ein Medium, bei dem einzelne Dokumente im Vordergrund standen.
Das klassische Web der ersten Hälfte des Jahrzehnts bestand aus Dokumenten mit Text, Bildern und Links.
Die zweite Hälfte machte das Hybrid Web aus.
Browsertechnologien, wie Javascript und CSS, sowie Plugins wie Flash und QuickTime wurden entwickelt und erweiterten die statischen Dokumente um interaktive Elemente.

In den 2000ern verbreitete sich das dynamische Senden und Anfordern von Daten unter dem Namen Ajax (Asynchronous JavaScript and XML).
Gegen Ende des Jahrzehnts wurden die vorhandenen Technologien genutzt, um ganze Anwendungen damit zu schreiben, wie z.~B. Facebook oder Twitter.

% section entwicklung_des_webs (end)


\section{Real-Time-Web} % (fold)
\label{sec:real-time-web}

Eine Real-Time-Web-Anwendung ist eine Web-Anwendung, die automatisch neue Daten an den Client sendet, sobald diese dem Server bekannt werden.
Dies unterscheidet sich vom klassischen Pull-Prinzip, in dem nur der Client per Anfrage eine neue Seite beim Server anfordern kann.
Bei Real-Time-Events ist es stattdessen notwendig, dass der Server mittels Push-Prinzip neue Informationen an den Client sendet \cite[vgl.][S.~1]{real_time_web}.
Facebook beispielsweise zeigt neue Statusmeldungen von Freunden an, ohne dass manuell die Seite neu geladen werden muss.

% section real-time-web (end)


\section{Motivation und Ziel} % (fold)
\label{sec:motivation_und_ziel}

Wenn man die in Abschnitt~\ref{sec:entwicklung_des_webs} beschriebene Entwicklung des Webs betrachtet, knüpft das Real-Time-Web genau daran an.
Der Nachrichtenverkehr verläuft nicht mehr nur vom Client zum Server, sondern auch in umgekehrter Richtung vom Server zum Client.
Das Play-Framework wird von Typesafe, der Firma von Scalas Begründer Martin Odersky, unterstützt und wirbt damit, die passenden Werkzeuge zur Entwicklung moderner Web-Anwendungen zu besitzen \cite[vgl.][]{play}.
Ziel dieser Arbeit ist es, herauszufinden, wie mit Scala und Play diese Art von Anwendungen entwickelt werden kann.
Dazu sollen die dafür benötigten Kernkomponenten vorgestellt und analysiert werden, um damit schließlich eine eigene Real-Time-Web-Anwendung zu entwickeln.

% section motivation_und_ziel (end)


\section{Themenabgrenzung} % (fold)
\label{sec:themenabgrenzung}

Diese Arbeit beschäftigt sich in erster Linie mit den Teilen des Play-Frameworks, die benötigt werden, um Real-Time-Web-Anwendungen damit zu entwickeln.
Zusätzlich wird benötigtes Hintergrundwissen, wie Architektur und Grundlagen des Frameworks zur Verfügung gestellt.
Einige Themenbereiche, die dafür nicht elementar sind, wie z.~B. Datenbanken, werden nicht behandelt.
Des Weiteren wird vorausgesetzt, dass der/die LeserIn bereits mit den Grundtechnologien der Web-Entwicklung, wie HTML, CSS und JavaScript vertraut ist.
Ein JavaScript-Framework wird aber nicht verwendet, um kein weiteres Vorwissen vorauszusetzen.

% section themenabgrenzung (end)


\section{Struktur} % (fold)
\label{sec:struktur}

Im \hyperref[cha:grundlagen]{zweiten Kapitel} wird gezeigt, welche grundlegenden Werkzeuge das Framwework für die Entwicklung von Web-Anwendungen besitzt und wie diese zu verwenden sind.
Dazu wird zunächst erklärt, wie eine statische Website erstellt wird, um diese später um Real-Time-Komponenten zu erweitern.
Nachdem die Grundtechniken zur Handhabung des Frameworks aufgezeigt worden sind, wird im \hyperref[cha:reaktive_programmierung]{dritten Kapitel} in die reaktive Programmierung eingeführt.
Diese ist grundlegend, um Plays Stream-Bibliothek zu verstehen, die im Mittelpunkt der Entwicklung von Real-Time-Web-Anwendungen steht.
Das \hyperref[cha:real_time_web]{vierte Kapitel} erklärt, wie auf der Client-Seite mittels JavaScript mit dem Server kommuniziert wird.
Mit diesem Wissen wird im \hyperref[cha:anwendung]{fünften Kapitel} schließlich eine Anwendung erstellt, die all die behandelten Punkte in praktischer Art und Weise zusammenführt.

% section struktur (end)


% chapter einleitung (end)