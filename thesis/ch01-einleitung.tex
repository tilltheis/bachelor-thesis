%!TEX root = thesis.tex

\chapter{Einleitung} % (fold)
\label{cha:einleitung}

In diesem Kapitel wird erläutert, was das Real-Time-Web ausmacht und weshalb sich das Play-Framework besonders eignet, Anwendungen dafür zu schreiben.
Es wird ein Überblick über die bisherige Entwicklung des Webs gegeben und klargestellt, womit sich diese Arbeit genau beschäftigt und womit nicht.
Anschließend wird aufgelistet, wie die weitere Struktur dieser Arbeit aussieht.


\section{Entwicklung des Webs} % (fold)
\label{sec:entwicklung_des_webs}

Nach \citealt{tavalsaari2011}~(S.~1--3) war das World Wide Web der 1990er Jahre ein Medium, bei dem einzelne Dokumente im Vordergrund standen.
Das klassische Web der ersten Hälfte des Jahrzehnts bestand aus Dokumenten mit Text, Bildern und Links.
Die zweite Hälfte machte das Hybrid Web aus.
Browsertechnologien, wie Javascript und CSS, sowie Plugins wie Flash und QuickTime wurden entwickelt und erweiterten die statischen Dokumente um interaktive Elemente.

In den 2000ern verbreitete sich das dynamische Senden und Anfordern von Daten unter dem Namen Ajax (Asynchronous JavaScript and XML).
Gegen Ende des Jahrzehnts wurden die vorhandenen Technologien genutzt, um ganze Anwendungen damit zu schreiben, wie z.~B. Facebook oder Twitter.

% section entwicklung_des_webs (end)


\section{Real-Time-Web} % (fold)
\label{sec:real-time-web}

Eine Real-Time-Web-Anwendung ist eine Web-Anwendung, die automatisch neue Daten an den Client sendet, sobald diese dem Server bekannt werden.
Dies unterscheidet sich vom klassischen Pull-Prinzip, in dem nur der Client per Anfrage eine neue Seite beim Server anfordern kann.
Bei Real-Time-Events ist es stattdessen notwendig, dass der Server mittels Push-Prinzip neue Informationen an den Client sendet \cite[vgl.][S.~1]{bozdag2007}.
Facebook beispielsweise zeigt neue Statusmeldungen von Freunden an, ohne dass manuell die Seite neu geladen werden muss.

% section real-time-web (end)


\section{Motivation} % (fold)
\label{sec:motivation}

Wenn man die in Kap.~\ref{sec:entwicklung_des_webs} beschriebene Entwicklung des Webs betrachtet, knüpft das Real-Time-Web genau daran an.
Der Nachrichtenverkehr verläuft nicht mehr nur vom Client zum Server, sondern auch in umgekehrter Richtung vom Server zum Client.
Das Play-Framework besitzt Werkzeuge, mit dieser Art der Kommunikation umzugehen~\cite[vgl.][S.~5]{drobi2012}.

% section motivation (end)


\section{Themenabgrenzung} % (fold)
\label{sec:themenabgrenzung}

Diese Arbeit beschäftigt sich in erster Linie mit der Architektur und den Grundlagen des Play-Frameworks und den Teilen, die zur Entwicklung von Real-Time-Web-Anwendungen benötigt werden.
Einige Themenbereiche, die dafür nicht elementar sind, wie z.~B. Datenbanken, werden nicht behandelt.
Des Weiteren wird vorausgesetzt, dass der Leser bereits mit den Grundtechnologien der Web-Entwicklung, wie HTML, CSS und Javascript vertraut ist.

% section themenabgrenzung (end)


\section{Ziel} % (fold)
\label{sec:ziel}

Ziel dieser Arbeit ist es, zu zeigen, wie mit dem Play-Framework Real-Time-Web-Anwendungen entwickelt werden können.
Es soll gezeigt werden, welche Werkzeuge das Framwework dafür besitzt und wie diese zu verwenden sind.
Dazu wird zunächst erklärt, wie eine statische Website erstellt wird, um diese später um Real-Time-Komponenten zu erweitern.
Nachdem die Grundtechniken zur Handhabung des Frameworks aufgezeigt worden sind, soll in die reaktive Programmierung eingeführt werden.
Diese ist grundlegend, um Plays Stream-Bibliothek zu verstehen, die im Mittelpunkt der Entwicklung von Real-Time-Web-Anwendungen steht.
Mit diesem Wissen soll schließlich eine Anwendung erstellt werden, die all die behandelten Punkte in praktischer Art und Weise zusammenführt.

% section ziel (end)


\section{Struktur} % (fold)
\label{sec:struktur}

% section struktur (end)


% chapter einleitung (end)