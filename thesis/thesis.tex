\documentclass[draft=false
              ,paper=a4
              ,twoside=false
              ,fontsize=11pt
              ,headsepline
              ,BCOR10mm
              ,DIV11
              ]{scrbook}
\usepackage[ngerman,english]{babel}
%% see http://www.tex.ac.uk/cgi-bin/texfaq2html?label=uselmfonts
\usepackage[T1]{fontenc}
\usepackage[utf8]{inputenc}
% \usepackage[latin1]{inputenc}
\usepackage{libertine}
\usepackage{pifont}
\usepackage{microtype}
\usepackage{textcomp}
\usepackage[german,refpage]{nomencl}
\usepackage{setspace}
\usepackage{makeidx}
\usepackage{listings}
\usepackage{natbib}
\usepackage[ngerman,colorlinks=true]{hyperref}
\usepackage{soul}
\usepackage{hawstyle}
\usepackage{lipsum} %% for sample text
\usepackage{scrhack} %% remove warning http://tex.stackexchange.com/questions/51867/koma-warning-about-toc
\usepackage{todonotes}

\bibpunct[, ]{(}{)}{,}{a}{}{;} % correct citation format (no comma before year)

\setcounter{secnumdepth}{5} % numbering for subsubsections


%% define some colors
\colorlet{BackgroundColor}{gray!20}
\colorlet{KeywordColor}{blue}
\colorlet{CommentColor}{black!60}
%% for tables
\colorlet{HeadColor}{gray!60}
\colorlet{Color1}{blue!10}
\colorlet{Color2}{white}

%% configure colors
\HAWifprinter{
  \colorlet{BackgroundColor}{gray!20}
  \colorlet{KeywordColor}{black}
  \colorlet{CommentColor}{gray}
  % for tables
  \colorlet{HeadColor}{gray!60}
  \colorlet{Color1}{gray!40}
  \colorlet{Color2}{white}
}{}
\lstset{%
  numbers=left,
  numberstyle=\tiny,
  stepnumber=1,
  numbersep=5pt,
  basicstyle=\ttfamily\small,
  keywordstyle=\color{KeywordColor}\bfseries,
  identifierstyle=\color{black},
  commentstyle=\color{CommentColor},
  backgroundcolor=\color{BackgroundColor},
  captionpos=b,
  fontadjust=true
}
\lstset{escapeinside={(*@}{@*)}, % used to enter latex code inside listings
        morekeywords={uint32_t, int32_t}
}
\ifpdfoutput{
  \hypersetup{bookmarksopen=false,bookmarksnumbered,linktocpage}
}{}

% "define" Scala http://tex.stackexchange.com/questions/47175/scala-support-in-listings-package
\lstdefinelanguage{scala}{
  morekeywords={abstract,case,catch,class,def,%
    do,else,extends,false,final,finally,%
    for,if,implicit,import,match,mixin,%
    new,null,object,override,package,%
    private,protected,requires,return,sealed,%
    super,this,throw,trait,true,try,%
    type,val,var,while,with,yield},
  otherkeywords={=>,<-,<\%,<:,>:,\#,@},
  sensitive=true,
  morecomment=[l]{//},
  morecomment=[n]{/*}{*/},
  morestring=[b]",
  morestring=[b]',
  morestring=[b]"""
}

%% more fancy C++
\DeclareRobustCommand{\cxx}{C\raisebox{0.25ex}{{\scriptsize +\kern-0.25ex +}}}

\clubpenalty=10000
\widowpenalty=10000
\displaywidowpenalty=10000

% unknown hyphenations
\hyphenation{
}

%% recalculate text area
\typearea[current]{last}

\makeindex
\makenomenclature

\begin{document}
\selectlanguage{ngerman}
\lstset{language=Scala}

%%%%%
%% customize (see readme.pdf for supported values)
\HAWThesisProperties{Author={Till Theis}
                    ,Title={Das Play-Framework und dessen Einsatz zur Entwicklung von Real-Time-Web-Anwendungen}
                    ,EnglishTitle={The Play Framework and its Use for Developing Real-Time Web Applications}
                    ,ThesisType={Bachelorarbeit}
                    ,ExaminationType={Bachelorprüfung}
                    ,DegreeProgramme={Bachelor of Science Angewandte Informatik}
                    ,ThesisExperts={Prof.
Dr.
Friedrich Esser \and Prof.
Dr.
Zweitprüfer}
                    ,ReleaseDate={1.
Januar 2345}
                  }

%% title
\frontmatter

%% output title page
\maketitle

\onehalfspacing

%% add abstract pages
%% note: this is one command on multiple lines
\HAWAbstractPage
%% German abstract
{Schlüsselwort 1, Schlüsselwort 2}%
{Dieses Dokument \ldots}
%% English abstract
{keyword 1, keyword 2}%
{This document \ldots}

\newpage
\singlespacing

\tableofcontents
\newpage
%% enable if these lists should be shown on their own page
%%\listoftables
%%\listoffigures
\lstlistoflistings

%% main
\mainmatter
\onehalfspacing
%% write to the log/stdout
\typeout{===== File: chapter 1}
%% include chapter file (chapter1.tex)
%%\include{chapter1}

%%%%


\chapter{Einleitung} % (fold)
\label{cha:einleitung}

In Diesem Kapitel wird erläutert, was das Real-Time-Web ausmacht und weshalb sich das Play-Framework besonders eignet, Anwendungen dafür zu schreiben.
Es wird ein Überblick über die bisherige Entwicklung des Webs gegeben und klargestellt, womit sich diese Arbeit genau beschäftigt und womit nicht.
Anschließend wird aufgelistet, wie die weitere Struktur dieser Arbeit aussieht.


\section{Entwicklung des Webs} % (fold)
\label{sec:entwicklung_des_webs}

Nach \citealt{tavalsaari2011}~(S.~1--3) war das World Wide Web der 1990er Jahre ein Medium, bei dem einzelne Dokumente im Vordergrund standen.
Das klassische Web der ersten Hälfte des Jahrzehnts bestand aus Dokumenten mit Text, Bildern und Links.
Die zweite Hälfte machte das hybrid Web aus.
Browsertechnologien, wie Javascript und CSS, sowie Plugins wie Flash und QuickTime wurden entwickelt und erweiterten die statischen Dokumente um interaktive Elemente.

In den 2000ern verbreitete sich das dynamische Senden und Anfordern von Daten unter dem Namen Ajax (Asynchronous JavaScript and XML).
Gegen Ende des Jahrzehnts wurden die vorhandenen Technologien genutzt, um ganze Anwendungen damit zu schreiben, wie z.~B. Facebook oder Twitter.

% section entwicklung_des_webs (end)


\section{Real-Time-Web} % (fold)
\label{sec:real-time-web}

Eine Real-Time-Web-Anwendung ist eine Web-Anwendung, die automatisch neue Daten an den Client sendet, sobald diese dem Server bekannt werden.
Dies unterscheidet sich vom klassischen Pull-Prinzip, in dem nur der Client per Anfrage eine neue Seite beim Server anfordern kann.
Bei Real-Time-Events ist es stattdessen notwendig, dass der Server mittels Push-Prinzip neue Informationen an den Client sendet \cite[vgl.][S.~1]{bozdag2007}.
Facebook beispielsweise zeigt neue Statusmeldungen von Freunden an, ohne dass manuell die Seite neu geladen werden muss.

% section real-time-web (end)


\section{Motivation} % (fold)
\label{sec:motivation}

Wenn man die in Kap.~\ref{sec:entwicklung_des_webs} beschriebene Entwicklung des Webs betrachtet, knüpft das Real-Time-Web genau daran an.
Der Nachrichtenverkehr verläuft nicht mehr nur vom Client zum Server, sondern auch in umgekehrter Richtung vom Server zum Client.
Das Play-Framework besitzt Werkzeuge, mit dieser Art der Kommunikation umzugehen~\cite[vgl.][S.~5]{drobi2012}.

% section motivation (end)


\section{Themenabgrenzung} % (fold)
\label{sec:themenabgrenzung}

Diese Arbeit beschäftigt sich in erster Linie mit der Architektur und den Grundlagen des Play-Frameworks und den Teilen, die zur Entwicklung von Real-Time-Web-Anwendungen benötigt werden.
Einige Themenbereiche, die dafür nicht elementar sind, wie z.~B. Datenbanken, werden nicht behandelt.
Des Weiteren wird vorausgesetzt, dass der Leser bereits mit den Grundtechnologien der Web-Entwicklung, wie HTML, CSS und Javascript vertraut ist.

% section themenabgrenzung (end)


\section{Ziel} % (fold)
\label{sec:ziel}

Das Ziel dieser Arbeit ist es, das Play-Framework in Scala zu analysieren und dessen Grundlagen zu erklären, um schließlich seine Werkzeuge zur Entwicklung von Real-Time-Web-Anwendungen zu erforschen.
Auf Basis dieser Erkenntnisse wird anschließend eine Beispielanwendung analysiert und eine eigene Anwendung geschrieben, die das Real-Time-Prinzip ausnutzt.

% section ziel (end)


% chapter einleitung (end)


\section{Struktur} % (fold)
\label{sec:struktur}

% section struktur (end)




\chapter{Grundlagen} % (fold)
\label{cha:grundlagen}

Grundlagen mit den bisherigen Quellen \citealt{hilton2013}.


\section{Vorbereitung} % (fold)
\label{sec:vorbereitung}

\subsection{Entwicklungsumgebung} % (fold)
\label{sub:entwicklungsumgebung}

Es kann entweder mit einem einfachen Texteditor entwickelt werden, oder auch mit einer IDE (Integrated development environment).

Beispiel Sublime Text 2.

Als IDE wird hier IntelliJ IDEA 12 Community Edition verwendet.
Nach dem Anlegen des Play-Projekts auf der Kommandozeile muss ein mal der Befehl \lstinline|play idea| ausgeführt werden.
Wenn man sich bereits in der Play-Eingabemaske befindet, muss man nur \lstinline|idea| ausführen.
Eine umfangreiche Liste für weitere IDEs ist bei \citealt{ide}~\footnote{\url{http://www.playframework.com/documentation/2.1.1/IDE}} zu finden.

% subsection entwicklungsumgebung (end)

% section vorbereitung (end)


% chapter grundlagen (end)




\chapter{Reaktive Programmierung} % (fold)
\label{cha:reaktive_programmierung}

Reaktive Programmierung ist die Programmierung von reaktiven System.
Ein reaktives System ist ein System, das von seiner Umgebung kontinuierlich Daten empfängt und darauf reagiert.
In nebenläufigen Systemen können gleich mehrere solcher Datenströme unabhängig voneinander existieren \cite[vgl.][S.~1]{reactive_programming}.

Reaktive Prozesse haben die Eigenschaft, schrittweise, immer wenn sie neue Daten empfangen, ein Ergebnis aufzubauen.
Deshalb sind sie bei Berechnungen essentiell, bei denen nicht nur das Endergebnis wichtig ist, sondern auch die Schritte, die dazu geführt haben \cite[vgl.][S.~2]{reactive_programming1}.


\section{Futures und Promises} % (fold)
\label{sec:futures_und_promises}

Futures und Promises mit den Quellen \citealt{haller2013} und \citealt{typesafe2013}.

% section futures_und_promises (end)


\section{Streams} % (fold)
\label{sec:streams}

Streams sind inkrementelle Datenströme, die nicht-blockierendes Lesen erlauben und kombiniert und transformiert werden können.
Sie erlauben inkrementelle, deklarative und funktionale Datenverarbeitung mit voller Kontrolle über den Ressourcenverbrauch.
Anwendungsbeispiele für Streams sind beispielsweise die Implementierung von Web-Servern oder Datenkompression.
Die in diesem Kapitel vorgestellten Streams sind auch bekannt als Iteratee I/O.
Sie entstanden im Umfeld der Programmiersprache Haskell und wurden von Oleg Kiselyov vorgestellt \cite[vgl.][S.~19]{monad_reader}.
Als Quellen hierfür dienen \citealt{kiselyov2012}, \citealt{iteratee_io}, \citealt{monad_reader} und die Seiten der offiziellen Play-Dokumentation (\citealt{iteratees}, \cite{enumerators}).


\subsection{Design} % (fold)
\label{sub:design}

Iteratee IO-Streams bestehen aus vier Komponenten.
\lstinline|Input|, \lstinline|Iteratee|, \lstinline|Enumerator| und \lstinline|Enumeratee|.
\lstinline|Input| beinhaltet die gestreamten Elemente.
\lstinline|Iteratee| konsumiert \lstinline|Input|-Elemente und dient somit als Datensenke.
\lstinline|Enumerator| generiert \lstinline|Input|-Elemente und dient somit als Datenquelle.
Und \lstinline|Enumeratee| transformiert einen \lstinline|Iteratee|.
Diese Komponenten und ihr jeweiliger Verwendungszweck wird in diesem Kapitel vorgestellt.


\subsubsection{Inputs} % (fold)
\label{ssub:design_inputs}

Die von der Datenquelle generierten Elemente werden in \lstinline|Input|s verpackt.
Der Datentyp \lstinline|Input| ist definiert, wie in Listing~\ref{lst:der_input_datentyp} gezeigt.

\begin{lstlisting}[caption=Der Input-Datentyp, label=lst:der_input_datentyp]
sealed trait Input[+E]
object Input {
  case class El[+E](e: E) extends Input[E]
  case object Empty extends Input[Nothing]
  case object EOF extends Input[Nothing]
}
\end{lstlisting}

Der Datentyp ist über den kovarianten Typ \lstinline|E| der gehaltenen Elemente parametrisiert.
Kovarianz bedeutet, dass wenn \lstinline|F| ein spezialisierter Typ von \lstinline|E| ist, \lstinline|Input[F]| auch ein spezialisierter Typ von \lstinline|Input[E]| ist \cite[vgl.][S.~11]{variance}.
Dadurch wären beispielsweise sowohl \lstinline|Input[Cat]|, also auch \lstinline|Input[Dog]| Spezialisierungen von \lstinline|Input[Animal]|.
Wäre \lstinline|Input| nicht kovariant über den Typ der der gehaltenen Elemente, wären \lstinline|Input[Cat]|, \lstinline|Input[Dog]| und \lstinline|Input[Animal]| völlig unterschiedliche Typen.

Des Weiteren ist anzumerken, dass \lstinline|Input| ein algebraischer Datentyp ist.
Ein algebraischer Datentyp ist ein Datentyp mit mehreren, alternativen Konstruktoren, die jeweils eigene Felder besitzen.
Mittels Pattern Matching kann von einem Wert eines solchen Datentyps auf den verwendeten Konstruktor geschlossen werden \cite[vgl.][S.~14--15]{algebraic_data_type}.

In Scala werden algebraische Datentypen mit Hilfe von einem Marker-Trait und \lstinline|case|-Klassen und -Objekten implementiert.
Das Marker-Trait ist der algebraische Datentyp und die \lstinline|case|-Klassen und -Objekte sind die Konstruktoren.
Die Nutzung von \lstinline|case|-Klassen und -Objekten hat den Vorteil, dass diese mittels Pattern Matching zerlegt werden können.
Dadurch, dass das Marker-Trait \lstinline|sealed| ist, wird verhindert, dass von außerhalb der Quellcode-Datei weitere Konstruktoren hinzugefügt werden können \cite[vgl.][]{algebraic_data_type_scala}.
In diesem Fall besitzt der algebraische Datentyp \lstinline|Input| die Konstruktoren \lstinline|El|, \lstinline|Empty| und \lstinline|EOF|.
Der Konstruktor \lstinline|El| besitzt ein Feld, \lstinline|Empty| und \lstinline|EOF| besitzen keine Felder.

Für den Fall, dass die Datenquelle nicht erschöpft ist, existiert im Companion-Objekt die Unterklasse \lstinline|Input.El| mit dazugehöriger Konstruktormethode.
Diese Implementierung hält genau ein Element aus der Datenquelle.
\lstinline|Input.Empty| "`signifies a stream with no immediately available data but which is still continuing"' \cite[vgl.][]{iteratee_io}\todo{Ist Übernehmen von Kiselyovs Definition für Plays Implementierung erlaubt?}.
Außerdem findet es im Kontext von \lstinline|Iteratee|s Verwendung, um zu signalisieren, dass kein Teil Eingabe übriggebliebenen ist.
Um zu signalisieren, dass die Datenquelle erschöpft ist, existiert das Objekt \lstinline|Input.EOF|.

Sowohl \lstinline|Input.Empty|, als auch \lstinline|Input.EOF| sind vom Typ \lstinline|Input[Nothing]|.
\lstinline|Nothing| ist ein besonderer Typ in Scala, der eine Spezialisierung jedes anderen Typs ist \cite[vgl.][S.~32]{scala_specification}.
Weil der Typparameter des \lstinline|Input|-Datentyps kovariant ist, lassen sich \lstinline|Input.Empty| und \lstinline|Input.EOF| mit jedem anderen \lstinline|Input| kombinieren.

Ein Wert vom Typ \lstinline|Input| hat also drei mögliche Zustände:
\begin{enumerate}
  \item Es gibt ein neues Element (\lstinline|Input.El|).
  \item Es gibt noch kein neues Element, doch die Datenquelle ist noch aktiv (\lstinline|Input.Empty|)
  \item Die Datenquelle ist erschöpft (\lstinline|Input.EOF|).
\end{enumerate}

% subsubsection design_inputs (end)


\subsubsection{Iteratees} % (fold)
\label{ssub:design_iteratees}

Die von der Datenquelle in \lstinline|Input|s verpackten Elemente werden von \lstinline|Iteratee|s konsumiert, um daraus ein Ergebnis aufzubauen.
\lstinline|Iteratee|s kapseln ihren Zustand \lstinline|Step|, der den Verarbeitungszustand widerspiegelt.
Der Typ des \lstinline|Step|-Zustands ist wie in Listing~\ref{lst:der_step_datentyp} gezeigt.
\begin{lstlisting}[caption=Der Step-Datentyp, label=lst:der_step_datentyp]
sealed trait Step[E, +A]
object Step {
  case class Done[+A, E](a: A, remaining: Input[E])
    extends Step[E, A]
  case class Cont[E, +A](k: Input[E] => Iteratee[E, A])
    extends Step[E, A]
  case class Error[E](msg: String, input: Input[E])
    extends Step[E, Nothing]
}
\end{lstlisting}

Der Typ ist über die zwei Typen \lstinline|E| und \lstinline|A| parametrisiert.
\lstinline|E| ist der Typ der zu konsumierenden Elemente.
\lstinline|A| ist der Typ des zu berechnenden Ergebnisses.

Ein \lstinline|Iteratee|, der mit seiner Berechnung fertig ist, ist im Zustand \lstinline|Step.Done|.
Ein solcher \lstinline|Iteratee| hält sein berechnetes Ergebnis und den Teil der letzten Eingabe, der nicht mehr verarbeitet wurde.
Ein \lstinline|Iteratee| in diesem Zustand nimmt keine weiteren Elemente an.
Häufig ist der Eingaberest \lstinline|Input.Empty|, weil die gesamte Eingabe konsumiert wurde.
Ein Beispiel für einen nicht leeren Eingaberest ist, wenn ein \lstinline|Iteratee| zeilenweise Dateiinhalte empfängt und darin nach einem Wort sucht.
Wenn das gesuchte Wort gefunden wurde, wird als Restwert der bisher nicht angesehene Rest der Zeile verwendet.

Ein \lstinline|Iteratee|, der noch kein Endergebnis berechnet hat, ist im Zustand \lstinline|Step.Cont|.
Das Argument von \lstinline|Step.Cont| ist die Schritt-Funktion \lstinline[breaklines=true]|Input[E] => Iteratee[E, A]|\todo{Continuation? Weil nicht genau nach Definition}.
Die Eingabe der Schritt-Funktion ist ein \lstinline|Input| mit einem neuen zu verarbeitenden Element.
Die Ausgabe der Schritt-Funktion ist ein neuer \lstinline|Iteratee|, der den neuen Berechnungsstand nach dem Verarbeiten des Eingabeelements hält.

Falls bei der Verarbeitung eines Elements ein Fehler auftritt, wird dies durch den Zustand \lstinline|Step.Error| signalisiert.
Ein solcher Fehler wird durch eine Fehlermeldung und durch das verursachende Eingabeelement dargestellt.
Beispielsweise kann ein \lstinline|Iteratee| nach dem Empfang eines \lstinline|Input.EOF|-Elements signalisieren, dass es noch mehr Daten benötigt.
%In diesem Fall kann das System versuchen doch noch mehr Daten zu finden, sodass der \lstinline|Iteratee| erfolgreich zu einem Ergebnis kommen kann.

Ein \lstinline|Iteratee| hat also einen Verarbeitungszustand vom Typ \lstinline|Step| mit drei möglichen Zuständen:
\begin{enumerate}
  \item Es wurde erfolgreich ein Ergebnis berechnet (\lstinline|Step.Done|).
  \item Es werden weitere Daten zur Berechnung benötigt (\lstinline|Step.Cont|).
  \item Es ist ein Fehler aufgetreten (\lstinline|Step.Error|).
\end{enumerate}

Außerdem ist anzumerken, dass der \lstinline|Iteratee|-Typ ist eine Monade.
Dadurch ist es möglich, einen beliebigen Wert in die Monade zu heben, die dann im \lstinline|Step.Done|-Zustand ist.
Auch kann ein erfolgreich berechneter Wert innerhalb der Monade transformiert werden.

% subsubsection design_iteratees (end)


\subsubsection{Enumerators} % (fold)
\label{ssub:design_enumerators}

Ein \lstinline|Enumerator| ist die Datenquelle, die ihre Daten, in \lstinline|Input|s verpackt, bereitstellt.
Der Datentyp \lstinline|Enumerator| ist definiert, wie in Listing~\ref{lst:der_enumerator_datentyp} vereinfacht dargestellt.
\begin{lstlisting}[caption=Der Enumerator-Datentyp, label=lst:der_enumerator_datentyp]
trait Enumerator[E] {
  def apply[A](i: Iteratee[E, A]): Future[Iteratee[E, A]]
}
\end{lstlisting}

Ein \lstinline|Enumerator| ist eine Funktion von \lstinline|Iteratee| nach \lstinline|Iteratee| und ist über die Typen des übergebenen \lstinline|Iteratee|s parametrisiert.
Der Eingabewert ist der \lstinline|Iteratee|, an den die generierten Elemente gesendet werden.
Der Rückgabewert ist der übergebene \lstinline|Iteratee| nach Fertigstellung oder Abbruch seiner Berechnung.
Der zurückgegebene \lstinline|Iteratee| befindet sich in der \lstinline|Future|-Monade.

Bei der Anwendung dieser Funktion werden dem \lstinline|Iteratee| so lange Daten übergeben, bis die Datenquelle erschöpft ist, oder bis der \lstinline|Iteratee| keine Daten mehr annimmt.
Die Herkunft der Daten, die der \lstinline|Enumerator| an den \lstinline|Iteratee| sendet, ist beliebig.
Weil der Rückgabetyp sich in der \lstinline|Future|-Monade befindet können auch zeitintensive Berechnungen durchgeführt werden, ohne den Programmfluss zu unterbrechen.
Dadurch ist es beispielsweise möglich, die Daten aus Dateien oder dem Netzwerk zu empfangen.

% subsubsection design_enumerators (end)


\subsubsection{Enumeratees} % (fold)
\label{ssub:design_enumeratees}

Ein \lstinline|Enumeratee| ist ein Stream-Transformator, der \lstinline|Iteratee|s eines bestimmten Element-Typs zu \lstinline|Iteratee|s eines anderen Element-Typs konvertiert.
Der Datentyp \lstinline|Enumeratee| ist definiert, wie in Listing~\ref{lst:der_enumeratee_datentyp} vereinfacht dargestellt.
\begin{lstlisting}[caption=Der Enumeratee-Datentyp, label=lst:der_enumeratee_datentyp]
trait Enumeratee[From, To] {
  def apply[A](inner: Iteratee[To, A]):
    Iteratee[From, Iteratee[To, A]]
}
\end{lstlisting}

Ein \lstinline|Enumeratee| ist eine Funktion von \lstinline|Iteratee| nach \lstinline|Iteratee|.
Der Eingabewert ist der zu transformierende \lstinline|Iteratee| vom inneren Element-Typ \lstinline|To|.
Der Ausgabewert ist ein \lstinline|Iteratee| vom äußeren Element-Typ \lstinline|From|.
Der neue \lstinline|Iteratee| nimmt Elemente vom Typ \lstinline|From| an und transformiert sie nach \lstinline|To|.
Diese transformierten Werte werden dann an den ursprünglichen \lstinline|Iteratee| weitergegeben.

Elemente vom Typ \lstinline|From| heißen äußere Elemente, weil sie vom resultierenden \lstinline|Iteratee| zuerst empfangen werden.
Elemente vom Typ \lstinline|To| heißen innere Elemente, weil sie zum \lstinline|Iteratee| gehören, das sich im Rückgabetyp des resultierenden \lstinline|Iteratee|s befindet.
Die Elemente des äußeren \lstinline|Iteratees| werden nach der Transformation an den inneren \lstinline|Iteratee| weitergereicht.

Die Transformation muss dabei aber nicht ein Element nach genau einem anderen Element abbilden.
Ein äußeres Element \lstinline|From| kann zu einem, keinem oder auch mehreren inneren Elementen \lstinline|To| abgebildet werden.
Genauso ist es möglich, dass mehrere äußere Elemente zu einem inneren Element zusammengefasst werden.

Ein \lstinline|Enumeratee| kann auch als \lstinline|Enumerator| betrachtet werden.
Dies kommt daher, dass \lstinline|Iteratee|s Monaden sind.
Ein \lstinline|Enumeratee| ist ein \lstinline|Enumerator|, dessen Ergebnis nicht in der \lstinline|Future|-Monade, sondern in der \lstinline|Iteratee|-Monade liegt.
Jeder \lstinline|Enumeratee| ist auch ein \lstinline|Iteratee|, wenn man seinen Rückgabewert betrachtet, der vom Typ \lstinline|Iteratee| ist.
Ein \lstinline|Enumeratee| ist also sowohl in der Rolle eines \lstinline|Iteratee|s, als auch in der Rolle eines \lstinline|Enumerator|s.
Er ist ein \lstinline|Iteratee| des äußeren Typs \lstinline|From| und ein \lstinline|Enumerator| des inneren Typs \lstinline|To|, weil er Elemente des äußeren Typs konsumiert und Elemente des inneren Typs generiert.

% subsubsection design_enumeratees (end)


\subsubsection{Komposition} % (fold)
\label{ssub:komposition}

\lstinline|Iteratee|s und \lstinline|Enumerator|s können sequentiell, parallel oder auch frei komponiert werden.
Bei der Komposition werden zwei oder mehr \lstinline|Iteratee|s oder \lstinline|Enumerator|s zu einem neuen \lstinline|Iteratee| oder \lstinline|Enumerator| kombiniert.

\paragraph{Komposition von Iteratees} % (fold)
\label{par:komposition_von_iteratees}\mbox{} % force new line

Sequenzielle Komposition zweier \lstinline|Iteratee|s funktioniert noch folgendem Prinzip:
Zuerst wird der erste \lstinline|Iteratee| angewendet, bis er keine Elemente mehr annimmt oder die Datenquelle erschöpft ist.
Dann wird der zweite \lstinline|Iteratee| auf die noch übrigen Elemente der Datenquelle angewendet, bis auch dieser \lstinline|Iteratee| keine Elemente mehr annimmt, oder die Datenquelle erschöpft ist.
Anschließend werden beide \lstinline|Iteratee|s kombiniert, z.~B. als Paar. % Iteratee#map/flatMap

Bei der parallelen Komposition werden die Elemente der Datenquelle immer an beide \lstinline|Iteratee|s weitergegeben.
Dies geschieht so lange, bis beide \lstinline|Iteratee|s keine weiteren Elemente mehr annehmen.
Anschließend werden beide \lstinline|Iteratee|s kombiniert. % Enumeratee.zip/zipWith

\lstinline|Iteratee|s könnnen auch frei kombiniert werden, sodass z.~B. nur bestimmte Elemente an den einen \lstinline|Iteratee| delegiert werden und alle anderen Elemente an den anderen \lstinline|Iteratee| delegiert werden.

% paragraph komposition_von_iteratees (end)

\paragraph{Komposition von Enumerators} % (fold)
\label{par:komposition_von_enumerators}\mbox{} % force new line

Die sequentielle Komposition zweier \lstinline|Enumerator|s erfolgt ähnlich, wie die sequentielle Kompositionen von \lstinline|Iteratee|s.
Der resultierende \lstinline|Enumerator| generiert erst alle Elemente des ersten \lstinline|Enumerator|s und dann alle Elemente des zweiten \lstinline|Enumerator|s.
Die sequentielle Komposition von \lstinline|Enumerator|s entspricht also der Verkettung ihrer Ausgaben. %Enumerator.andThen

Parallele Komposition von \lstinline|Enumerator|s bedeutet, dass jedes Element des kombinierten \lstinline|Enumerator|s entweder aus dem ersten, oder aus dem zweiten \lstinline|Enumerator| stammt.
Die Reihenfolge der Elemente ist dadurch bestimmt, welcher \lstinline|Enumerator| zuerst ein neues Element zur Verfügung stellt. % Enumerator.interleave

\lstinline|Enumerator|s können auch frei kombiniert werden, wobei beliebig gewählt werden kann, aus welchem der zur Verfügung stehenden \lstinline|Enumerator|s gelesen wird.
Dabei ist es durchaus denkbar, dass Elemente mehrerer \lstinline|Enumerator|s zu einem Element kombiniert werden.
Auch ist es möglich, dass in Abhängigkeit der generierten Elemente eines \lstinline|Enumerator|s, ein anderer \lstinline|Enumerator| für das nächste Element ausgesucht wird.
Beispielsweise kann solange aus dem einen \lstinline|Enumerator| gelesen werden, bis ein bestimmtes Element auftritt.
Dann wird der andere \lstinline|Enumerator| ausgewählt, aus dem die folgenden Elemente extrahiert werden.

% paragraph komposition_von_enumerators (end)

% subsubsection komposition (end)


% subsection design (end)


\subsection{Anwendung} % (fold)
\label{sub:anwendung}

Im vorigen Kapitel wurden die einzelnen Komponenten vorgestellt.
In diesem Kapitel soll der ihr Einsatz mit Hilfe von konkreten Code-Beispielen gezeigt werden.


\subsubsection{Iteratees} % (fold)
\label{sssec:iteratees}

Plays \lstinline|Iteratee|s bilden einen Container um ihren \lstinline|Step|-Zustand.
Dadurch werden, wie bei objektorientierten Sprachen üblich\todo{Zitat}, Daten und Operationen gebündelt.
Die elementare Methode dieser Klasse ist \lstinline|fold|.
Die Signatur dazu ist in Listing~\ref{lst:fold_signatur} zu lesen.
\begin{lstlisting}[caption=Die Signatur von fold, label=lst:fold_signatur]
def fold[B](folder: Step[E, A] => Future[B]): Future[B]
\end{lstlisting}

Mit Hilfe von \lstinline|fold| lässt sich der Zustand des \lstinline|Iteratee|s transformieren.
Deshalb sind sehr viele Methoden in der \lstinline|Iteratee|-Klasse auf Basis von \lstinline|fold| definiert.
Durch Angabe der Schrittfunktion im \lstinline|Cont|-Konstruktor bestimmt die konkrete \lstinline|Iteratee|-Implementierung, wie mit übergebenen Elementen umzugehen ist.
Nur durch diese Funktion können neue Elemente gelesen werden.
Dadurch wird der zu berechnende Wert aufgebaut.

\paragraph{Iteratees erstellen} % (fold)
\label{par:iteratees_erstellen}\mbox{} % force new line

Es gibt drei Möglichkeiten, einen neuen \lstinline|Iteratee| zu erstellen.
\begin{enumerate}
  \item Durch Erstellung einer neuen Klasse, die von \lstinline|Iteratee| erbt und \lstinline|fold| implementiert.
  \item Durch Benutzung einer Konstruktormethode analog zu den \lstinline|Step|-Zuständen (\lstinline|Done|, \lstinline|Cont| und \lstinline|Error|).
  \item Durch Benutzung einer Hilfsfunktion des Companion-Objekts (\lstinline|fold|, \lstinline|foreach|, u.~a.).
\end{enumerate}

Zu Demonstrationszwecken soll nun ein \lstinline|Iteratee| erstellt werden, der alle Elemente aufsummiert.
Es wird dabei nach jeder der drei möglichen Varianten implementiert.

\subparagraph{Erstellung durch Vererbung} % (fold)
\label{subp:erstellung_durch_vererbung}\mbox{} % force new line

Die erste Variante ist, von \lstinline|Iteratee| zu erben und die \lstinline|fold|-Methode zu implementieren.
Dies erfordert viel Schreibarbeit, weil nur die sehr generische \lstinline|fold|-Methode verwendet werden kann.
Der Code dazu ist in Listing~\ref{lst:iteratee_durch_vererbung} zu sehen.

\begin{lstlisting}[caption=Erstellung eines Iteratees durch Vererbung, label=lst:iteratee_durch_vererbung]
case class SumIteratee(sum: Int = 0) extends Iteratee[Int, Int] {
  def fold[B](folder: Step[Int, Int] => Future[B]): Future[B] = {
    folder(Step.Cont(_ match {
      case Input.El(i) => SumIteratee(sum + i)
      case Input.Empty => this
      case Input.EOF   => new Iteratee[Int, Int] {
        def fold[B](folder: Step[Int, Int] => Future[B]) = {
          folder(Step.Done(sum, Input.EOF))
        }
      }
    }))
  }
}

val sumIterateeFromInheritance: Iteratee[Int, Int] = SumIteratee()
\end{lstlisting}

Der \lstinline|folder|-Funktion muss der aktuelle \lstinline|Step|-Zustand übergeben werden.
Diese Funktion soll den \lstinline|Iteratee| dann ausführen, d.~h. ihn mit Elementen versorgen.
Anschließend soll der \lstinline|folder| dann aus dem finalen Zustand ein Ergebnis berechnen.
Solange noch weitere Elemente kommen können, also solange kein \lstinline|Input.EOF|-Element empfangen wurde, geht der \lstinline|Iteratee| in einen neuen \lstinline|Cont|-Zustand über.
Bei diesem Zustandsübergang wird auch die interne Berechnung fortgesetzt.
Es wird also in einen \lstinline|Cont|-Zustand übergegangen, der die aktualisierte Summe hält.
Wenn ein leeres Element (\lstinline|Input.Empty|) verarbeitet werden soll, wird der \lstinline|Iteratee| unverändert zurückgegeben, weil sich an der Gesamtsumme nichts geändert hat.
Sobald keine weiteren Elemente mehr kommen können, wird die Berechnung beendet und ein \lstinline|Iteratee| an den \lstinline|folder| übergeben, der sich immer im \lstinline|Done|-Zustand befindet.

% subparagraph erstellung_durch_vererbung (end)

\subparagraph{Erstellung durch Konstruktormethode} % (fold)
\label{subp:erstellung_durch_konstruktormethode}\mbox{} % force new line

Die zweite Möglichkeit, einen \lstinline|Iteratee| zu erstellen, ist mit Hilfe einer der Konstruktormethoden.
Es gibt die Konstruktormethoden \lstinline|Done|, \lstinline|Cont| und \lstinline|Error|, die jeweils einen \lstinline|Iteratee| im gleichnamigen \lstinline|Step|-Zustand erstellen.
Der Code, der einen \lstinline|Iteratee| mit Hilfe dieser Methoden erstellt, ist in Listing~\ref{lst:iteratee_durch_konstruktormethode} zu sehen.

\begin{lstlisting}[caption=Erstellung eines Iteratees durch eine Konstruktormethode, label=lst:iteratee_durch_konstruktormethode]
def sumIteratee(sum: Int = 0): Iteratee[Int, Int] = Cont(_ match {
  case Input.El(i) => sumIteratee(sum + i)
  case Input.Empty => sumIteratee(sum)
  case Input.EOF   => Done(sum, Input.EOF)
})

val sumIterateeFromConstructor: Iteratee[Int, Int] = sumIteratee()
\end{lstlisting}

Der Code ist dem in \nameref{subp:erstellung_durch_vererbung} vorgestellten Code sehr ähnlich.
Im Gegensatz dazu muss hier jedoch nicht mehr explizit von \lstinline|Iteratee| geerbt und die \lstinline|fold|-Methode implementiert werden.
Dies macht den Code schon wesentlich kürzer und einfacher.
Der Code ist auf die elementaren Aufgaben reduziert und es ist nun klar zu erkennen, was bei welcher Eingabe geschieht.

% subparagraph erstellung_durch_konstruktormethode (end)

\subparagraph{Erstellung durch Hilfsmethode} % (fold)
\label{subp:erstellung_durch_hilfsmethode}\mbox{} % force new line

Die dritte und letzte Möglichkeit nutzt eine problemspezifische Hilfsmethode.
Das Aufsummieren von Menge von Zahlen lässt sich sehr einfach mit Hilfe eines Folds implementieren\todo{Erklären, was ein Fold ist}.
Das \lstinline|Iteratee|-Companion-Objekt beinhaltet u.~a. die \lstinline|fold|-Hilfsmethode, die für solche Operationen gedacht ist.
Durch Einsatz dieser Methode wird die Lösung, wie in Listing~\ref{lst:iteratee_durch_hilfsmethode} zu erkennen, zum Einzeiler.

\begin{lstlisting}[caption=Erstellung eines Iteratees durch eine Hilfsmethode, label=lst:iteratee_durch_hilfsmethode]
val sumIterateeFromHelper: Iteratee[Int, Int] =
  Iteratee.fold(0)(_ + _)
\end{lstlisting}

Vom Filtern der unterschiedlichen \lstinline|Input|-Typen ist hierbei abstrahiert und es muss sich nur noch um tatsächliche Eingabeelemente gekümmert werden.

% subparagraph erstellung_durch_hilfsmethode (end)

% paragraph iteratees_erstellen (end)

\paragraph{Iteratees ausführen} % (fold)
\label{par:iteratees_ausfuehren}\mbox{} % force new line

Um die \lstinline|Iteratee|s mit Daten zu versorgen, existiert, wie in \nameref{par:iteratees_erstellen} beschrieben, die Methode \lstinline|fold|.
\lstinline|fold| nimmt als Argument eine Funktion \lstinline|folder| vom Typ \lstinline[breaklines=true]|Step[E, A] => Future[B]|.
Diese Funktion übergibt ein Element an den \lstinline|Iteratee| und muss daraufhin den Zustand des neuen \lstinline|Iteratee|s überprüfen, ob weitere Elemente akzeptiert werden, oder nicht nicht.
Erst nach dieser Analyse kann entschieden werden, ob ein weiteres Element übergeben wird oder nicht.
In Listing~\ref{lst:iterateeausfuehrung_durch_folder} wird gezeigt, wie dies implementiert werden kann.

\begin{lstlisting}[caption=Ausführung eines Iteratees durch folder-Funktion, label=lst:iterateeausfuehrung_durch_folder]
def folder(xs: Int*)(step: Step[Int, Int]): Future[Int] = {
  def folder_(xs: List[Int])(step: Step[Int, Int]): Future[Int] =
    xs match {
      case Nil => step match {
        case Step.Cont(k) => k(Input.EOF).fold(_ match {
          case Step.Done(sum, Input.EOF) => Future(sum)
          case _ => Future.failed(new Exception("invalid state"))
        })
        case _ => Future.failed(new Exception("invalid state"))
      }
      case x :: xs => step match {
        case Step.Cont(k) => k(Input.El(x)).fold(folder_(xs))
        case _ => Future.failed(new Exception("invalid state"))
      }
    }

  folder_(xs.toList)(step)
}

val sumResult: Future[Int] =
  sumIterateeFromHelper.fold(folder(1, 4, -2))
\end{lstlisting}

Das Übergeben von Daten mittels der \lstinline|fold|-Methode ist, wie in Listing~\ref{lst:iterateeausfuehrung_durch_folder} zu erkennen, recht aufwändig.
Durch Verwendung von \lstinline|Enumerator|s wird dies allerdings wesentlich einfacher.

% paragraph iteratees_ausfuehren (end)

% subsubsection iteratees (end)

\subsubsection{Enumerators} % (fold)
\label{ssub:enumerators}

Mit Hilfe von \lstinline|Enumerator|s können auf einfache Weise Daten an \lstinline|Iteratee|s übergeben werden.
In diesem Kapitel werden Anwendung und Funktionsweise von \lstinline|Enumerator|s gezeigt.
Es wird erklärt, auf welche Arten \lstinline|Enumerator|s erstellt werden können, und wie sie genutzt werden können, um Daten an \lstinline|Iteratee|s zu übergeben.

\paragraph{Enumerators erstellen} % (fold)
\label{par:enumerators_erstellen}\mbox{} % force new line

Es gibt drei unterschiedliche Techniken, einen \lstinline|Enumerator| zu erstellen.
\begin{enumerate}
  \item Durch Erstellung einer neuen Klasse, die von \lstinline|Enumerator| erbt und \lstinline|apply| implementiert.
  \item Durch Benutzung einer Konstruktormethode im Companion-Objekt (\lstinline|apply|, \lstinline|generateM|, u.~a.).
  \item Durch Benutzung einer Konstruktormethode im \lstinline|Concurrent|-Objekt (\lstinline|broadcast|, \lstinline|unicast|, u.~a.).
\end{enumerate}

Im Folgenden werden alle Herangehensweisen vorgestellt, um Zahlen für den im vorigen Kapitel vorgestellten Summierungsiteratee zu generieren.

\subparagraph{Erstellung durch Vererbung} % (fold)
\label{subp:enumeratorerstellung_durch_vererbung}\mbox{} % force new line

Bei der Erstellung eines \lstinline|Enumerator|s durch Vererbung, muss das \lstinline|Enumerator|-Trait implementiert werden.
Alles, was dafür nötig ist, ist die \lstinline|apply|-Methode zu implementieren.
Innerhalb dieser Methode wird der übergebene \lstinline|Iteratee| mit Hilfe seiner \lstinline|fold|-Methode befüllt, wie in Listing~\ref{lst:enumeratorerstellung_durch_vererbung} zu sehen.

\begin{lstlisting}[caption=Erstellung eines Enumerators durch Vererbung, label=lst:enumeratorerstellung_durch_vererbung]
case class NumberEnumerator(xs: Int*) extends Enumerator[Int] {
  def apply[A](iteratee: Iteratee[Int, A]):
      Future[Iteratee[Int, A]] = {
    xs.foldLeft(Future(iteratee)) { (futureIteratee, x) =>
      futureIteratee.flatMap { iteratee =>
        iteratee.fold {
          case Step.Cont(k) => Future(k(Input.El(x)))
          case _ => Future(iteratee)
        }
      }
    }
  }
}

val numberEnumeratorFromInheritance: Enumerator[Int] =
  NumberEnumerator(1, 4, -2)
\end{lstlisting}

Alles was ein \lstinline|Enumerator| zu tun hat, ist seine Daten an den übergebenen \lstinline|Iteratee| zu übergeben.
Dieses Vorgehen ist in diesem Beispiel als Fold über den \lstinline|Iteratee| implementiert.
Es wird immer ein Element an den \lstinline|Iteratee| übergeben und der resultierende \lstinline|Iteratee| als neues Zwischenergebnis verwendet.
Falls der \lstinline|Iteratee| keine weiteren Elemente akzeptiert, obwohl noch Elemente übrig sind, wird dieser als Endergebnis verwendet.
Das hier vorgestellte Verfahren ist in leicht abgewandelter Form der Implementierung von \lstinline|Enumerator.apply| entnommen \cite[vgl.][Zeile~663 und Zeile~690]{play_enumerator_source_code}.

% subparagraph enumeratorerstellung_durch_vererbung (end)

\subparagraph{Erstellung durch Konstruktormethode im Companion-Objekt} % (fold)
\label{subp:erstellung_durch_konstruktormethode_im_companion-object}\mbox{} % force new line

Die zweite Variante, einen \lstinline|Enumerator| zu erstellen, ist mit Hilfe einer der Konstruktormethoden des Companion-Objekts.
Hierfür gibt es mehrere hilfreiche Methoden, die an dieser Stelle aber nicht alle vorgestellt werden können.
Stattdessen wird exemplarisch eine Methode vorgestellt, die auch in den folgenden Beispielen verwendet werden wird.

Die hier vorgestellte Methode ist die \lstinline|apply|-Methode des Companion-Objekts mit der Signatur \lstinline[breaklines=true]|def apply[E](in: E*): Enumerator[E]|.
Diese Methode nimmt als Parameter beliebig viele Elemente eines Typs und wird diese bei Anwendung in \lstinline|Input|-Elemente verpacken und an den \lstinline|Iteratee| übergeben.
Wie der \lstinline|Enumerator| mit dieser Methode erstellt wird, ist in Listing~\ref{lst:enumeratorerstellung_durch_apply} zu sehen.

\begin{lstlisting}[caption=Erstellung eines Enumerators durch die apply-Konstruktormethode, label=lst:enumeratorerstellung_durch_apply]
val numberEnumeratorFromApply: Enumerator[Int] =
  Enumerator(1, 4, -2)
\end{lstlisting}

% subparagraph erstellung_durch_konstruktormethode_im_companion-object (end)

\subparagraph{Erstellung durch Konstruktormethode im Concurrent-Objekt} % (fold)
\label{subp:erstellung_durch_konstruktormethode_im_concurrent_objekt}\mbox{} % force new line

Das Objekt \lstinline|Concurrent| stellt weitere Konstruktoren für \lstinline|Enumerator|s bereit.
Unter anderem beinhaltet dieses Objekt die Methode \lstinline|unicast|, die imperatives schreiben auf einen \lstinline|Iteratee| erlaubt.
Die Signatur von \lstinline|unicast| ist wie in Listing~\ref{lst:unicast_signatur} dargestellt.

\begin{lstlisting}[caption=Die Signatur von Concurrent.unicast, label=lst:unicast_signatur]
def unicast[E](
  onStart: (Channel[E]) => Unit,
  onComplete: => Unit,
  onError: (String, Input[E]) => Unit): Enumerator[E]
\end{lstlisting}

Die Methode nimmt als Argumente drei Funktionen: \lstinline|onStart|, \lstinline|onComplete| und \lstinline|onError|.
\lstinline|onStart| wird immer aufgerufen, wenn die \lstinline|apply|-Methode des \lstinline|Enumerator|s aufgerufen wird.
\lstinline|onComplete| wird aufgerufen, wenn der \lstinline|Iteratee| in den \lstinline|Step.Done|-Zustand übergegangen ist.
\lstinline|onError| wird aufgerufen, wenn der \lstinline|Iteratee| in den \lstinline|Step.Error|-Zustand übergegangen ist.
Die letzten beiden Funktionen sind optionale Argumente.

Die \lstinline|onStart|-Funktion nimmt ein Argument vom Typ \lstinline|Channel|.
Ein \lstinline|Channel| hat u.~a. die Methoden \lstinline[breaklines=true]|def push(item: E): Unit| und \lstinline[breaklines=true]|def end(): Unit|.
Diese Methoden erlauben es, ein Element an das verbundene \lstinline|Iteratee| zu übergeben, bzw. zu signalisieren, dass der \lstinline|Enumerator| keine weiteren Elemente mehr hat.

Um mit Hilfe der \lstinline|unicast|-Methode einen Zahlen generierenden \lstinline|Enumerator| zu erstellen, wird nur die \lstinline|onStart|-Funktion benötigt.
Diese Funktion ist dafür zuständig, die Elemente an den \lstinline|Channel| und somit an den verbundenen \lstinline|Iteratee| zu übergeben.
Wichtig ist dabei jedoch, dass nach Übergabe aller Zahlen der \lstinline|Channel| wieder geschlossen wird.
Andernfalls wird niemals signalisiert, dass keine weiteren Elemente mehr kommen und der \lstinline|Iteratee| wird nicht in den \lstinline|Step.Done|-Zustand übergehen.
Die Implementation dazu ist in Listing~\ref{lst:enumeratorerstellung_durch_unicast} zu sehen.

\begin{lstlisting}[caption=Erstellung eines Enumerators durch die unicast-Konstruktormethode, label=lst:enumeratorerstellung_durch_unicast]
def numberEnumerator(xs: Int*): Enumerator[Int] = {
  Concurrent.unicast { channel =>
    xs.foreach(channel.push(_))
    channel.end
  }
}

val numberEnumeratorFromUnicast: Enumerator[Int] =
      numberEnumerator(1, 4, -2)
\end{lstlisting}

% subparagraph erstellung_durch_konstruktormethode_im_concurrent_objekt (end)

% paragraph enumerators_erstellen (end)

\paragraph{Anwendung auf Iteratees} % (fold)
\label{par:anwendung_auf_iteratees}\mbox{} % force new line

\lstinline|Enumerator|s sind, wie in Kap.~\ref{ssub:design_enumerators} erklärt, Funktionen.
Das Ergebnis der Anwendung einer solchen \lstinline|Enumerator|-Funktion ist der übergebene \lstinline|Iteratee| nach Konsum der Elemente in der \lstinline|Future|-Monade.
Ein Beispiel für diese ersten Schritte ist in Listing~\ref{lst:enumeratoranwendung1} zu sehen.

\begin{lstlisting}[caption=Anwendung eines Enumerators auf einen Iteratee, label=lst:enumeratoranwendung1]
val iteratee: Iteratee[Int, Int] = Iteratee.fold(0)(_ + _)
val enumerator: Enumerator[Int] = Enumerator(1, 4, -2)
val futureIterateeAfterApplication: Future[Iteratee[Int, Int]] =
  enumerator(iteratee)
\end{lstlisting}


Um mit dem neuen \lstinline|Iteratee| weiterzuarbeiten wäre es möglich, alle weiteren Operationen innerhalb der \lstinline|Future|-Monade durchzuführen.
Weil es aber mühsam ist, \lstinline|Iteratee|s innerhalb von \lstinline|Future|s zu bearbeiten, ist es möglich einen Wert vom Typ \lstinline|Future[Iteratee[E, A]]| in einen Wert vom Typ \lstinline|Iteratee[E, A]| zu transformieren.
Beide Typen haben die gleiche Bedeutung, indem sie für ein Ergebnis stehen, das möglicherweise noch nicht vorhanden ist.
Weil ein \lstinline|Iteratee| innerhalb der \lstinline|Future|-Monade aus diesem Betrachtungswinkel aber redundant ist, existiert die Methode \lstinline|Iteratee.flatten|.


Nachdem ein \lstinline|Enumerator| alle Elemente an einen \lstinline|Iteratee| übergeben hat, sendet er allerdings kein \lstinline|Input.EOF|-Element.
Dadurch ist es möglich, mehrere \lstinline|Enumerator|s auf einen \lstinline|Iteratee| anzuwenden, worauf in Kap.~\ref{par:anwendung_komposition_von_enumerators} näher eingegangen wird.
Um ein \lstinline|Input.EOF|-Element an ein \lstinline|Iteratee| zu senden, wird die \lstinline|run|-Methode aus dem \lstinline|Iteratee|-Trait aufgerufen.
Eine Weiterführung des zuvor begonnenen \lstinline|Enumerator|-Beispiels ist in Listing~\ref{lst:enumeratoranwendung2} zu finden.
In diesem Beispiel werden die beiden oben beschriebenen Techniken angewendet, um die berechnete Summe aus dem \lstinline|Iteratee| zu extrahieren.

\begin{lstlisting}[caption=Extrahierung des Ergebnisses aus einem Iteratee, label=lst:enumeratoranwendung2]
val iterateeAfterApplication: Iteratee[Int, Int] =
  Iteratee.flatten(futureIterateeAfterApplication)

val futureResult: Future[Int] = iterateeAfterApplication.run
\end{lstlisting}

Mit Hilfe der Methode \lstinline|Enumerator.run| ist es möglich, \lstinline|Enumerator|anwendung und Ergebnisextraktion zusammenzufassen.
Diese Hilfsmethode hat die Signatur \lstinline[breaklines=true]|def run[A](i: Iteratee[E, A]): Future[A]|.
Unter Einsatz dieser Methode wird die Verarbeitung im Aufsummierungsbeispiel, wie in Listing~\ref{lst:enumeratoranwendung3} zu sehen, auf einen Befehl verkürzt.

\begin{lstlisting}[caption=Anwendung eines Enumerators mit gleichzeitiger Ergebnisextrahierung, label=lst:enumeratoranwendung3]
val futureResult2: Future[Int] = enumerator.run(iteratee)
\end{lstlisting}

% paragraph anwendung_auf_iteratees (end)

% subsubsection enumerators (end)


\subsubsection{Enumeratees} % (fold)
\label{ssub:enumeratees}

\lstinline|Enumeratee|s sind Stream-Transformatoren, die \lstinline|Iteratee|s eines Typs zu \lstinline|Iteratee|s eines anderen Typs transformieren.
Sie sind, wie in Kap.~\ref{ssub:design_enumeratees} erklärt, Funktionen von \lstinline|Iteratee| nach \lstinline|Iteratee|.
Das Ergebnis des resultierenden \lstinline|Iteratee|s ist wiederum ein \lstinline|Iteratee|.
Und zwar ist es der ursprünglichen \lstinline|Iteratee| nach Empfang der Elemente.
In diesem Kapitel wird beschrieben, wie...

\paragraph{Erstellung von Enumeratees} % (fold)
\label{par:erstellung_von_enumeratees}\mbox{} % force new line

Es gibt drei Möglichkeiten, einen \lstinline|Enumeratee| zu erstellen.

\begin{enumerate}
  \item Durch Erstellung einer neuen Klasse, die das \lstinline|Enumeratee|-Trait implementiert.
  \item Durch Benutzung einer Konstruktormethode des Companion-Objekts (\lstinline|map|, \lstinline|filter|, u.~a.)
  \item Durch Benutzung einer Konstruktormethode des \lstinline|Traversable|-Objekts (\lstinline|take|, \lstinline|drop|, u.~a.)
\end{enumerate}

Im Folgenden werden die ersten beiden Möglichkeiten vorgestellt und mit Beispielen veranschaulicht.
In den Beispielen wird ein \lstinline|Enumeratee| implementiert, der die Zahlen eines \lstinline|Iteratee[Int, Int]| mit 2 multipliziert.
Die dritte Möglichkeit ist an dieser Stelle nur der Vollständigkeit halber aufgeführt, weil sie vom Autor als zu speziell erachtet wird.
Es sei hier nur erwähnt, dass das \lstinline|Traversable|-Objekt einige wenige Konstruktormethoden für \lstinline|Enumeratee|s bereitstellt, die auf Elementen vom Typ \lstinline|TraversableLike| aus der Scala-Standardbibliothek arbeiten.

\subparagraph{Erstellung durch Vererbung} % (fold)
\label{subp:enumerateeerstellung_durch_vererbung}\mbox{} % force new line

Das \lstinline|Enumeratee|-Trait verlangt, dass die Methode \lstinline|applyOn| implementiert wird.
Die Methode \lstinline|apply| ist als Alias für \lstinline|applyOn| definiert.
\lstinline|applyOn| hat die Signatur, die in Kap.~\ref{ssub:design_enumeratees} für \lstinline|apply| vorgestellt wurde.
Eine Mögliche Implementierung für einen \lstinline|Enumeratee|, der die Zahlen eines \lstinline|Iteratee|s mit 2 multipliziert ist in Listing~\ref{lst:enumerateeerstellung_durch_vererbung} zu sehen.

\begin{lstlisting}[caption=Erstellung eines Enumeratees durch Vererbung, label=lst:enumerateeerstellung_durch_vererbung]
case object MultiplyingEnumeratee extends Enumeratee[Int, Int] {
  def applyOn[A](inner: Iteratee[Int, A]):
      Iteratee[Int, Iteratee[Int, A]] = {
    Iteratee.flatten(inner.fold {
      case Step.Cont(k) => Future(Cont {
        case Input.El(number) =>
          MultiplyingEnumeratee(k(Input.El(number * 2)))
        case Input.Empty => MultiplyingEnumeratee(k(Input.Empty))
        case Input.EOF => Done(Cont(k))
      })
      case _ => Future(Done(inner, Input.Empty))
    })
  }
}

val enumerateeFromInheritance: Enumeratee[Int, Int] =
  MultiplyingEnumeratee
\end{lstlisting}

Die Implementierung dieses \lstinline|Enumeratee|s ist der von \lstinline|Enumeratee.map| nachempfunden.
Sie ist allerdings weniger abstrakt, als in der Referenzimplementierung \cite[vgl.][Zeile~268, 174, 81]{play_enumeratee_source_code}.
Der \lstinline|Enumeratee| prüft, ob der innere \lstinline|Iteratee| weitere Elemente annehmen kann.
Kann er keine weiteren Elemente annehmen, wird er als Ergebnis zurückgegeben.
Kann er Elemente annehmen, werden diese transformiert und anschließend weitergegeben.
Sobald ein \lstinline|Input.EOF| empfangen wird, wird in den Endzustand mit dem finalen inneren \lstinline|Iteratee| als Ergebnis übergegangen.

% subparagraph enumerateeerstellung_durch_vererbung (end)

\subparagraph{Erstellung durch Konstruktormethode des Companion-Objekts} % (fold)
\label{subp:enumerateeerstellung_durch_konstruktormethode_des_companion_objekts}\mbox{} % force new line

Die zweite Möglichkeit, einen \lstinline|Enumeratee| zu erstellen ist mit Hilfe seines Companion-Objekts.
Im Companion-Objekt finden sich mehrere Konstruktormethoden, die Listenoperationen sehr ähneln, wie z.~B. \lstinline|drop|, \lstinline|filter|, \lstinline|map| oder \lstinline|zip|.
Um alle Elemente des \lstinline|Iteratees| zu transformieren bietet sich \lstinline|map| an.
Mit Hilfe dieser Methode werden alle Elemente durch eine angegebene Funktion transformiert, bevor sie an das innere \lstinline|Iteratee| gegeben werden.
\lstinline|map| hat folgende Signatur:
\begin{lstlisting}
def map[E]: AnyRef { def apply[NE](f: E => NE): Enumeratee[E,NE] }
\end{lstlisting}
Diese Signatur wirkt sperrig, lässt sich aber sinngemäß auf folgende Vereinfachung reduzieren:
\begin{lstlisting}
def map[E, NE](f: E => NE): Enumeratee[E, NE]
\end{lstlisting}

Der wichtige Unterschied zwischen diesen beiden Signaturen ist, dass die erste nur über einen Typ \lstinline|E| parametrisiert ist.
Erst das zurückgegebene Objekt verlangt den zweiten Typ \lstinline|NE|.
Dadurch wird, so vermutet der Autor, Code, der diese Methode nutzt, kürzer, weil der Parameter \lstinline|NE| in der Regel vom Compiler abgeleitet werden kann.
So muss in aller Regel, sofern überhaupt nötig, nur noch der erste Typ-Parameter angegeben werden.
Zu einem früheren Entwicklungsstand des Frameworks wurde die zweite Variante verwendet und wurde erst später zur aktuellen Signatur abgeändert \cite[vgl.][]{play_enumeratee_map_signatur}.

Unter Einsatz dieser \lstinline|map|-Methode ist die Transformation in einer einzigen Anweisung möglich.
Der dazu nötige Code kann Listing~\ref{lst:enumerateeerstellung_durch_companion_konstruktormethode} entnommen werden.

\begin{lstlisting}[caption=Erstellung eines Enumeratees durch die map-Konstruktormethode, label=lst:enumerateeerstellung_durch_companion_konstruktormethode]
val enumerateeFromCompanion: Enumeratee[Int, Int] =
  Enumeratee.map(_ * 2)
\end{lstlisting}

% subparagraph enumerateeerstellung_durch_konstruktormethode_des_companion_objekts (end)

% paragraph erstellung_von_enumeratees (end)

% subsubsection enumeratees (end)


\subsubsection{Komposition} % (fold)
\label{ssub:anwendung_komposition}

Die in Kap.~\ref{ssub:komposition} vorgestellten Arten von Komposition sollen an dieser Stelle durch Anwendungsbeispiele veranschaulicht werden.
Es wird auf die Komposition von \lstinline|Iteratee|s und \lstinline|Enumerator|s eingegangen.

\paragraph{Komposition von Iteratees} % (fold)
\label{par:anwendung_komposition_von_iteratees}\mbox{} % force new line

Im Folgenden werden die drei Arten der Komposition von \lstinline|Iteratee|s vorgestellt.
Es werden die zugrundeliegenden Ansätze und die jeweilige Umsetzung erläutert.

\subparagraph{Sequentielle Komposition} % (fold)
\label{subp:anwendung_sequentielle_komposition_von_iteratees}\mbox{} % force new line

Zwei \lstinline|Iteratee|s sequentiell zu kombinieren, bedeutet zuerst den ersten \lstinline|Iteratee| und dann den zweiten \lstinline|Iteratee| auszuführen.
Dafür eignen sich die Methoden \lstinline|map| und \lstinline|flatMap| der \lstinline|Iteratee|-Klasse.
Diese Methoden erlauben es, wie bei jeder Monade, den Wert innerhalb der Monade zu transformieren.
Der Wert der \lstinline|Iteratee|-Monade steht dann fest, wenn der \lstinline|Iteratee| in den \lstinline|Done|-Zustand übergegangen ist.
Um zwei \lstinline|Iteratee| nacheinander auszuführen, wird bei der \lstinline|flatMap|-Operation auf dem ersten \lstinline|Iteratee| der zweite \lstinline|Iteratee| zurückgegeben und die Ergebnisse der beiden \lstinline|Iteratee|s anschließend kombiniert.
Ein Code-Beispiel, das dieses Prinzip ausnutzt, ist in Listing~\ref{lst:sequentielle_komposition_von_iteratees} zu lesen.

\begin{lstlisting}[caption=Sequentielle Komposition von Iteratees, label=lst:sequentielle_komposition_von_iteratees]
val i1: Iteratee[Int, Option[Int]] = Iteratee.head
val i2: Iteratee[Int, Int] = Iteratee.fold(0)(_ + _)

val i12: Iteratee[Int, (Option[Int], Int)] =
  i1.flatMap(res1 => i2.map(res2 => (res1, res2)))

val e: Enumerator[Int] = Enumerator(1, 4, -2)
val result: Future[(Option[Int], Int)] = e.run(i12)
// result = Future.successful((Some(1), 2))
\end{lstlisting}

In diesem Fall gibt es zwei \lstinline|Iteratee|s.
Der erste \lstinline|Iteratee| konsumiert genau ein Element und gibt dieses, in einem \lstinline|Option|-Wert verpackt, zurück.
Der \lstinline|Option|-Typ ist notwendig, damit auch ein Ergebnis existiert, wenn nur ein \lstinline|Input.EOF|, aber kein \lstinline|Input.El| empfangen wird.
Der zweite \lstinline|Iteratee| summiert alle empfangenen Elemente auf und gibt die Summe als Ergebnis zurück.
Das Ergebnis der Komposition dieser beiden \lstinline|Iteratee|s ist ein Paar aus dem Ergebnis des ersten und des zweiten \lstinline|Iteratee|s.
Das Kombinieren von \lstinline|i1| und \lstinline|i2| lässt sich durch das Verwenden einer \lstinline|for|-Comprehension allerdings noch lesbarer gestalten, wie in Listing~\ref{lst:sequentielle_komposition_von_iteratees_mit_for_comprehension} zu sehen.

\begin{lstlisting}[caption=Sequentielle Komposition von Iteratees mit for-Comprehension, label=lst:sequentielle_komposition_von_iteratees_mit_for_comprehension]
val i12: Iteratee[Int, (Option[Int], Int)] = for {
  res1 <- i1
  res2 <- i2
} yield (res1, res2)
\end{lstlisting}

% subparagraph anwendung_sequentielle_komposition_von_iteratees (end)entieller oder paralleler Komposition zu halten.

\subparagraph{Parallele Komposition} % (fold)
\label{subp:anwendung_parallele_komposition_von_iteratees}\mbox{} % force new line

Bei der parallelen Komposition werden die Elemente immer an beide \lstinline|Iteratee|s übergeben.
Dies lässt sich mit Hilfe von \lstinline|Enumeratee|s und den Methoden \lstinline|zip|, bzw. \lstinline|zipWith| bewerkstelligen.
\lstinline|zip| erstellt aus zwei \lstinline|Iteratee|s einen neuen \lstinline|Iteratee|, der die empfangen Elemente an beide \lstinline|Iteratee|s weitergibt.
Sobald beide \lstinline|Iteratee|s im \lstinline|Done|-Zustand sind, werden die Ergebnisse als Paar zurückgegeben.
Die Verwendung von \lstinline|zipWith| ermöglicht es, das Ergebnis nicht nur als Paar zu erhalten, sondern es kann eine Funktion übergeben werden, die die Ergebnisse zu einem beliebigen Datentyp zusammenfasst.
In Listing~\ref{lst:parallele_komposition_von_iteratees} ist ein Beispiel zu sehen, das die beiden \lstinline|Iteratee|s aus dem Beispiel für \nameref{lst:sequentielle_komposition_von_iteratees} verwendet, sie aber parallel kombiniert.

\begin{lstlisting}[caption=Parallele Komposition von Iteratees, label=lst:parallele_komposition_von_iteratees]
val i1: Iteratee[Int, Option[Int]] = Iteratee.head
val i2: Iteratee[Int, Int] = Iteratee.fold(0)(_ + _)

val i12: Iteratee[Int, (Option[Int], Int)] =
  Enumeratee.zip(i1, i2)

val e: Enumerator[Int] = Enumerator(1, 4, -2)
val result: Future[(Option[Int], Int)] = e.run(i12)
// result = Future.successful((Some(1), 3))
\end{lstlisting}

% subparagraph anwendung_parallele_komposition_von_iteratees (end)

\subparagraph{Freie Komposition} % (fold)
\label{subp:anwendung_freie_komposition_von_iteratees}\mbox{} % force new line

\lstinline|Iteratees| können auch kombiniert werden, ohne sich an die Regeln von sequentieller oder paralleler Komposition zu halten.

% subparagraph anwendung_freie_komposition_von_iteratees (end)

% paragraph anwendung_komposition_von_iteratees (end)

\paragraph{Komposition von Enumerators} % (fold)
\label{par:anwendung_komposition_von_enumerators}\mbox{} % force new line

Genau wie \lstinline|Iteratee|s kombiniert werden können, so können auch \lstinline|Enumerator|s miteinander kombiniert werden.
Es existieren wieder drei verschiedene Kompositionsansätze, deren Umsetzung im Folgenden näher beschrieben wird.

\subparagraph{Sequentielle Komposition} % (fold)
\label{subp:anwendung_sequentielle_komposition_von_enumerators}\mbox{} % force new line

Sequentielle Komposition zweier \lstinline|Enumerator|s heißt, dass erst die Elemente des ersten \lstinline|Enumerator|s generiert werden, bis dieser erschöpft ist und anschließend die Elemente des zweiter \lstinline|Enumerator|s generiert werden.
Das \lstinline|Enumerator|-Trait stellt hierfür die Methode \lstinline|andThen| bereit.
Diese Methode erstellt einen neuen \lstinline|Enumerator|, der die Elemente des ursprünglichen \lstinline|Enumerator|s und dann die Elemente des übergebenen \lstinline|Enumerator|s enumeriert.
Listing~\ref{lst:sequentielle_komposition_von_enumerators} zeigt, wie \lstinline|andThen| verwendet werden kann, um zwei \lstinline|Enumerator|s sequentiell zu komponieren.

\begin{lstlisting}[caption=Sequentielle Komposition von Enumerators, label=lst:sequentielle_komposition_von_enumerators]
val i: Iteratee[Int, List[Int]] = Iteratee.getChunks
val e1 = Enumerator(1, 2)
val e2 = Enumerator(3)
val e12: Enumerator[Int] = e1.andThen(e2)

val result: Future[List[Int]] = e12.run(i)
// result = Future.successful(List(1, 2, 3))
\end{lstlisting}

% subparagraph anwendung_sequentielle_komposition_von_enumerators (end)

\subparagraph{Parallele Komposition} % (fold)
\label{subp:anwendung_parallele_komposition_von_enumerators}\mbox{} % force new line

Wenn immer das zuerst verfügbare Element zweier \lstinline|Enumerator|s generiert werden soll, wird dies als parallele Komposition bezeichnet.
Das \lstinline|Enumerator|-Companion-Objekt stellt hierfür die Konstruktormethode \lstinline|interleave| bereit.
\lstinline|interleave| kommt in unterschiedlichen Varianten, darunter auch in einer Variante mit der in Listing~\ref{lst:signatur_von_interleave} gezeigten Signatur.

\begin{lstlisting}[caption=Die Signatur von interleave, label=lst:signatur_von_interleave]
def interleave[E](e1: Enumerator[E], es: Enumerator[E]*):
  Enumerator[E]
\end{lstlisting}

\lstinline|interleave| nimmt einen oder mehrere \lstinline|Enumerator|s als Argumente und kombiniert diese nach dem zuvor beschriebenen Prinzip.
In Listing~\ref{lst:parallele_komposition_von_enumerators} wird nun mit Hilfe dieser Methode gezeigt, wie mehrere \lstinline|Enumerator|s parallel komponiert werden können.

\begin{lstlisting}[caption=Parallele Komposition von Enumerators, label=lst:parallele_komposition_von_enumerators]
def timeoutEnumerator[A](x: A, d: Duration): Enumerator[A] =
  Enumerator.flatten(Promise.timeout(Enumerator(x), d))

val i: Iteratee[Int, List[Int]] = Iteratee.getChunks
val e1 = timeoutEnumerator(1, 300 milliseconds)
val e2 = timeoutEnumerator(2, 100 milliseconds)
val e3 = timeoutEnumerator(3, 200 milliseconds)
val e123: Enumerator[Int] = Enumerator.interleave(e1, e2, e3)

val result: Future[List[Int]] = e123.run(i)
// result = Future.successful(List(2, 3, 1))
\end{lstlisting}

Zu Bemerken ist hierbei, dass \lstinline|Iteratee.getChunks| die empfangenen Elemente als Liste zurückgibt.
Außerdem stellt Play das Objekt \lstinline|play.api.libs.concurrent.Promise| mit mehreren Hilfsmethoden zur Verfügung.
Mit dessen \lstinline|timeout|-Methode ist es möglich, ein \lstinline|Future|-Objekt zu erstellen, dass nach Verlauf einer angegebenen Zeitspanne einen Wert liefert.
\lstinline|Enumerator.flatten| transformiert einen Wert vom Typ \lstinline|Future[Enumerator[A]]| nach \lstinline|Enumerator[A]| analog zur Methode \lstinline|Iteratee.flatten|, die in Kap.~\ref{par:anwendung_auf_iteratees} behandelt wurde.
Gut zu erkennen ist, dass die Reihenfolge der Elemente im Ergebnis mit der jeweiligen Zeitspanne korreliert und nicht etwa mit der Reihenfolge, in der die \lstinline|Enumerator|s an \lstinline|interleave| übergeben wurden.

% subparagraph anwendung_parallele_komposition_von_enumerators (end)

\subparagraph{Freie Komposition} % (fold)
\label{subp:anwendung_freie_komposition_von_enumerators}\mbox{} % force new line

Auch \lstinline|Enumerator|s können miteinander kombiniert werden, ohne die dass der resultierende \lstinline|Enumerator| die Eigenschaften von sequentieller oder paralleler Komposition hat.

% subparagraph anwendung_freie_komposition_von_enumerators (end)

% paragraph anwendung_komposition_von_enumerators (end)

% subsubsection anwendung_komposition (end)


\subsubsection{Regeln} % (fold)
\label{ssub:regeln}

Die folgenden Regeln sind \citealt{kiselyov2012}~(S.~12--13) entnommen und dienen derzeit nur zum Verständnis des Autors.
Sie werden später, sofern möglich, an geeigneter Stelle einsortiert.

\lstinline!k |> l! ist der Alternativ-Kombinator. Der Iteratee, der als erstes ein Ergebnis liefert, wird zurückgegeben.

\begin{enumerate}
  \item Komposition
    \begin{lstlisting}
enstr (s1 ++ s2) = enstr s2 . enstr s1
    \end{lstlisting}

    Eingaben konkatenieren ist das gleiche, wie Enumeratoren zu konkatenieren.

  \item Verkettung
    \begin{lstlisting}
enstr (s1 ++ s2)(i >>= f) = enstr s1 i >>= enstr s2 . f
    \end{lstlisting}

    Wenn Iteratee i s1 erkennt, dann ist das Verarbeiten und Verwerten der der konkatenierten Eingaben das gleiche, wie das Verarbeiten und Verwerten der ersten Eingabe und anschließend das Verarbeiten der zweiten Eingabe.

  \item Neutrales Element
    \begin{lstlisting}
failure >>= f = failure
    \end{lstlisting}

    Wenn ein Iteratee einen Fehler erzeugt, haben auch folgende Iteratees keine Chance mehr auf Erfolg.

  \item Rechtsdistributivität
    \begin{lstlisting}
i >>= \x -> (k1 x |> k2 x) = (i >>= k1) |> (i >>= k2)
    \end{lstlisting}

    Iteratees sind rechtsdistributiv, d.h. erst Iteratee i auszuführen und auf das Ergebnis k1 oder k2 auszuführen ist das gleiche, als i und darauf k1 auszuführen oder i und darauf k2 auszuführen.
\end{enumerate}

% subsubsection regeln (end)

% subsection anwendung (end)


% section streams (end)


% section serverseitig (end)

% chapter reaktive_programmierung (end)




\chapter{Schluss} % (fold)
\label{cha:schluss}

% chapter schluss (end)


%%%%

%% appendix if used
%%\appendix
%%\typeout{===== File: appendix}
%%\include{appendix}

% bibliography and other stuff
\backmatter

\typeout{===== Section: literature}
%% read the documentation for customizing the style
\bibliographystyle{dinat}
\bibliography{thesis}

\typeout{===== Section: nomenclature}
%% uncomment if a TOC entry is needed
%%\addcontentsline{toc}{chapter}{Glossar}
\renewcommand{\nomname}{Glossar}
\clearpage
\markboth{\nomname}{\nomname} %% see nomencl doc, page 9, section 4.1
\printnomenclature

%% index
\typeout{===== Section: index}
\printindex

\HAWasurency

\end{document}
